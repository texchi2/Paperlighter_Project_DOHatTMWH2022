\documentclass[12pt,twoside]{article}
\maxdeadcycles=1000 % Output loop---200 consecutive dead cycles.

\usepackage{paperlighter}

% Remove page numbering
%\pagenumbering{gobble}

\usepackage{longtable, setspace, outlines, tabularx}
 % for OliveGreen
\definecolor{OliveGreen}{rgb}{0,0.6,0}
% Set link colors
%\usepackage[rgb,dvipsnames]{xcolor}
%\hypersetup{colorlinks=true, linkcolor=RoyalBlue, urlcolor=RoyalBlue}

\usepackage{pdfpages} % for includepdf

\usepackage{selinput}
%\usepackage[margin=2cm]{geometry}
\usepackage{enumitem,varwidth}
%\usepackage[svgnames]{xcolor}
\usepackage{tikz} % for SWOT
\usetikzlibrary{shapes.geometric}

% Recommended, but optional, packages for figures and better typesetting:
\usepackage{microtype}
\usepackage{graphicx}
\usepackage{subfigure}
\usepackage{booktabs} % for professional tables

% Attempt to make hyperref and algorithmic work together better:
\newcommand{\theHalgorithm}{\arabic{algorithm}}


% For theorems and such
\usepackage{amsmath}
\usepackage{amssymb}
\usepackage{mathtools}
\usepackage{amsthm}

% if you use cleveref..
\usepackage[capitalize,noabbrev]{cleveref}

%%%%%%%%%%%%%%%%%%%%%%%%%%%%%%%%
% THEOREMS
%%%%%%%%%%%%%%%%%%%%%%%%%%%%%%%%
\theoremstyle{plain}
\newtheorem{theorem}{Theorem}[section]
\newtheorem{proposition}[theorem]{Proposition}
\newtheorem{lemma}[theorem]{Lemma}
\newtheorem{corollary}[theorem]{Corollary}
\theoremstyle{definition}
\newtheorem{definition}[theorem]{Definition}
\newtheorem{assumption}[theorem]{Assumption}
\theoremstyle{remark}
\newtheorem{remark}[theorem]{Remark}

% Todonotes is useful during development; simply uncomment the next line
%    and comment out the line below the next line to turn off comments
%\usepackage[disable,textsize=tiny]{todonotes}
\usepackage[textsize=tiny]{todonotes}

% Chinese
\usepackage{xeCJK} % for Chinese, compiling by XeLaTex
\usepackage{indentfirst}
\setlength{\parindent}{2em}  % setting the indentation to be two Chinese characters size.
\usepackage{fontspec} %設定字體
% Fandol font (the default)  not shown "內"
\setCJKmainfont{[Iansui094-Regular.ttf]}
\setCJKsansfont{[Iansui094-Regular.ttf]}
\setCJKmonofont{[Iansui094-Regular.ttf]}
%\setCJKmainfont{AR PL UMing TW MBE} % AR PL UMing TW MBE or "UKai" https://www.overleaf.com/learn/latex/Questions/Which_OTF_or_TTF_fonts_are_supported_via_fontspec%3F#Chinese
%BiauKai} %標楷體 from macOS %設定中文為系統上的字型,而英文不去更動,使用原TeX字型
%\setCJKmainfont[Vertical=RotatedGlyphs]{AR PL UMing TW MBE}


%%%%%%%%%%%%%
%% Chinese section numbering
%% by https://tex.stackexchange.com/questions/535650/customize-section-numbering-format-for-chinese-typesetting
%% answered Apr 1, 2020 at 13:10 wave
% https://tex.stackexchange.com/users/208733/wave

\usepackage{zhnumber}

\usepackage{titlesec}
\usepackage{titling}
%\usepackage{fontspec}
\usepackage{newunicodechar}
\usepackage{tocloft} % adding the tocloft package for toc customization

\setcounter{tocdepth}{5}
\setcounter{secnumdepth}{5}

% zhnum[style={Traditional,Financial}] doesn't work with the section counter,
% so we define our own counter and increase it every time in \thesection
\newcounter{mysec}[section]
\renewcommand\thesection{%
    \addtocounter{mysec}{1}%
    \zhnum[style={Traditional,digitals}]{mysec}、} % Financial
\renewcommand\thesubsection{(\zhnum{subsection})、} % added a 、
\renewcommand\thesubsubsection{(\zhnum{subsubsection})} % added parentheses
% (full-width, don't know if that's what you want)
\renewcommand\theparagraph{} % you don't want paragraph numbers
\renewcommand\thesubparagraph{} % nor subparagraph numbers

% we have to adjust the spacing in the toc because the section label is longer than usual
\addtolength\cftsecnumwidth{1em}
\addtolength\cftsubsecindent{1em}
\addtolength\cftsubsubsecindent{1em}

% here we need to make sure the normal section counter is accessed
\titleformat{\section}{\Large\bfseries\sffamily\filcenter}
    {\zhnum[style={Traditional,digitals}]{section}、}{.5em}{} % financial
% not really sure what you intend to achieve with \fontsize but I'll leave it here
\titleformat*{\subsection}{\bfseries\sffamily} %\fontsize{18}{20}
\titleformat*{\subsubsection}{\bfseries\sffamily} %\fontsize{16}{18}

% no extra version for numberless is necessary since no numbers are used anyways
% also you get newlines from omitting the [display] in \titleformat already
\titleformat{\paragraph}
    {\fontsize{14}{16}\bfseries\sffamily}{}{0em}{} 
\titleformat{\subparagraph}
    {\fontsize{12}{14}\bfseries\sffamily}{}{0em}{}
% we need the following so that they don't indent (second argument, 0em);
% you'll have to adjust the spacing though since this is not display style anymore:
\titlespacing*{\paragraph}{0em}{3.25ex plus 1ex minus .2ex}{.75ex plus .1ex} 
\titlespacing*{\subparagraph}{0em}{3.25ex plus 1ex minus .2ex}{.75ex plus .1ex}

\renewcommand{\maketitlehooka}{\sffamily}

\renewcommand{\baselinestretch}{1.2}
\renewcommand{\contentsname}{目錄} %{目次}

\usepackage{hyperref}
\hypersetup{
  colorlinks=true,
  linkcolor=[rgb]{0,0.37,0.53},
  citecolor=[rgb]{0,0.47,0.68},
  filecolor=[rgb]{0,0.37,0.53},
  urlcolor=[rgb]{0,0.37,0.53},
  % pagebackref=true, % this is ignored
  linktoc=all}

%\usepackage{hyperref}
% \usepackage[utf8x]{inputenc} do not use inputenc with XeTeX
% \usepackage{fixltx2e} not required any more

%\usepackage{rotating}

%%%%%%

\slimtitle{特需中心} %Paperlighter Example}
\slimauthor{萬芳醫院牙科部}


\begin{document} % 獎勵醫院提供特殊需求者牙科醫療服務
%衛生福利部
%111 年度「特殊需求者牙科醫療服務獎助計畫」 申請獎助作業規定
%本計畫經費財源為菸品健康福利捐,屬特定收入來源;年度進行中

%111年度「特殊需求者牙科醫療服務獎助計畫」計畫申請書

\thispagestyle{empty}

\lightertitle{
%\Large
\hspace{2.3cm} %臺北市立萬芳醫院牙科部
衛生福利部醫療發展基金獎助辦理

\hspace{0.6cm} 111年度「特殊需求者牙科醫療服務獎助計畫」

\hspace{5.6cm} 計畫申請書}
% 特殊需求者牙醫醫療服務
%居家牙醫醫療服務計畫
% 計畫申請書內容與格式
% 綜合資料表
% 進階照護院所:口腔醫療與保健推廣計畫書
\vspace{1.6cm}
\begin{figure}[H]
    \centering
\includegraphics[width=10cm]{MOHW_LOGO_0030101001.pdf}
\end{figure}
\vspace{2cm}

{\Large
計畫名稱: \underline{111年度「特殊需求者牙科醫療服務獎助計畫」}
\vspace{4mm}

執行單位: \underline{臺北市立萬芳醫院牙科部} %-委託財團法人私立臺北醫學大學辦理
\vspace{4mm}

計畫主持人: \underline{祁力行} \hspace{5cm} 職稱: \underline{特殊需求者口腔醫學科主任}
\vspace{1mm}

計畫聯絡人: \underline{張芝菱}
\vspace{4mm}

計畫執行期間: \underline{111年4月1日至12月31日}
}
\clearpage



%%
\lighterauthor{祁力行$^{\dagger}$, 李勝揚$^{\ddagger}$}

\lighteraddress{$^\dagger$}{臺北市立萬芳醫院牙科部口腔顎面外科}
\lighteraddress{$^\ddagger$}{臺北市立萬芳醫院牙科部}


\lighteremail{110255@w.tmu.edu.tw}
%110050@w.tmu.edu.tw}

% table of contents
\tableofcontents \clearpage

%\begin{abstract}
%摘要:主旨
臺北市立萬芳醫院牙科部設置特殊需求者牙科門診(以下稱特需中心),每週門診14診次,優先服務身心障礙者,牙科櫃台附有身心障礙服務窗口,設置志工,與負責身心障礙門診之牙科輔助人員共同幫忙引導、溝通、協助患者就醫。候診區及初診區均有輪椅無障礙空間。
%先申請特需中心,然後牙科部專任醫師才具有到宅牙醫服務的資格,當然也需要身障照護的學分。到宅小組需要一名醫師、一名護理師,攜帶必要設備,每一名案主收案都事先申請核可
特殊需求者的口腔照護,因日常生活打理由照顧者代勞,往往無法顧及基本的口腔衛生照護,在宅臥床患者更面臨齲齒、牙周病細菌感染的重大危機,因其就醫之阻礙甚多,四、五層樓之公寓往往沒有電梯。
特需中心配合長照2.0政策,規畫居家牙醫醫療服務。
%Using \LaTeX{} to write papers is concise and convenient. However, for writing in life, complicated \LaTeX{} style-files (e.g., elegantpaper) are difficult to access, or submission style-files (e.g., journal or conference) are not free indeed. To tackle these problems and satisfy an elegant and straightforward scientific writing, \textbf{paperlighter.sty}, a one-column style-file, is designed. This document is edited from icml2022.sty and provides a basic paper template. Compared to icml2022.sty, paperlighter.sty contain fewer operations, reducing adjustment while keep graceful. \textbf{\textit{Notably, the paper's main content only describes the format of icml2022.sty. We place the content to show the actual effect of paperlighter.sty.}}
\end{abstract} % no more
%\input{content/format}
%{
%一、計畫書封面:至少包含計畫名稱(包含計畫執行地區)、計畫執行單位、計 畫執行期間。
%二、書寫格式:以 word 建檔,A4 版面,由左而右,由上而下,標楷體 14 號字 型,橫式書寫,填寫綜合資料表,另計畫書編有目錄頁碼。

%三、計畫本文至少應包括: (一)緣起/前言:請敘述申請本計畫產生之背景。 
%(二)現況分析:請敘述現實施地區所呈現問題。 
%(三)計畫人力配置:組織架構、現況、醫事人力、醫療設備、經營現況;另詳述醫事人力(專任或兼任醫師、執業登錄)及參與獎助服務醫師名冊 並檢附受有 10 學分以上身心障礙相關教育訓練之證明文件(僅需列出指 定課程 10 個學分數,並完整確實標註清楚)。

%(四)計畫內容: 
%1、目的。
%2、計畫執行期程、申請之獎助項目及預估獎勵金額。
%3、依序撰寫各項預定辦理內容與進行步驟。
%4、過去 3 年計畫執行情形及未來 3 年執行目標、摘要表(如附件 1、6)、衛生局指定辦理身心障礙者特別門診之資格證明文件(如公文,並得為影、複本)。
%5、品質和成效。 
%6、預期效益。

%(五)後續發展或推廣:詳述計畫執行結束後之後續規劃,以達獎助後之永續責任,並可嘗試建立標準模式,提供其他地區標竿參考。

%}
% 口腔醫療與保健推廣及保健推廣計畫書
% 僅第一項: 補助醫院提供特殊需求者牙科醫療服務
\section{緣起}
臺北市立萬芳醫院牙科部設置特殊需求者牙科門診(以下稱特需中心),每週門診2診次,專門服務身心障礙者,牙科櫃台附有身心障礙服務窗口,設置志工,與負責身心障礙門診之牙科輔助人員共同幫忙引導、溝通、協助患者就醫。候診區及初診區均有輪椅無障礙空間。
%先申請特需中心,然後牙科部專任醫師才具有到宅牙醫服務的資格,當然也需要身障照護的學分。到宅小組需要一名醫師、一名護理師,攜帶必要設備,每一名案主收案都事先申請核可
特殊需求者的口腔照護,因日常生活打理由照顧者代勞,往往無法顧及基本的口腔衛生照護,行動不便、甚至長期在宅臥床的患者更面臨齲齒、牙周病細菌感染的重大危機,因其就醫之阻礙甚多,四、五層樓之公寓往往沒有電梯。口腔細菌感染有機會吸入造成肺炎感染,增加患者身體的負擔。
萬芳醫院牙科部特需中心,配合長照2.0政策,除服務原有的到院病患,更自111年3月起進一步籌備居家牙醫醫療服務(以下稱到宅牙醫)。

\section{現況分析}
身心障礙者特別門診,每週開設2診(週四下午陳培惠醫師、週六上午謝承祐醫師)。家庭牙醫科陳醫師,及兒童牙科謝醫師,均已接受30學分身心障礙教育訓練。除執行適當的口腔與牙齒治療,並對患者與其陪同家人進行口腔保健教育。110年共計服務人次300人,個案管理追蹤人數60名。
	
%「全民健康保險會」協定年度醫療給付費用總額事項」辦理
% https://www.nhi.gov.tw/BBS_Detail.aspx?n=73CEDFC921268679&sms=D6D5367550F18590&s=DC5108C6E3DD21D0

本院牙科部依據「111年全民健康保險牙醫門診總額特殊醫療服務計畫」,辦理「特殊需求者到宅牙醫計畫」,旨在擴大特定身心障礙者口腔醫療與保健推廣。預計每年可穩定服務(每名患者每2個月一次)20名以上的居家醫療需求者。
%(以下稱本計畫)。

\section{計畫人力配置}

\subsection{組織架構}
\begin{outline}

%\1 組織架構
\1 現況: 萬芳醫院牙科部,「特殊需求者口腔醫學科」籌備處,規畫整合原有特需門診,加上到宅牙醫服務,形成完整的特需服務與衛生教育推廣網絡
\1 醫事人力: %詳述醫事人力(專任或兼任醫師、執業登錄)及參與獎助服務醫師名冊並檢附受有 10 學分以上身心障礙相關教育訓練之證明文件(僅需列出指定課程10個學分數,並完整確實標註清楚)。
本院專任醫師三名,口腔顎面外科主任祁力行醫師、
家庭牙醫科主任陳培惠醫師、
兒童牙科主任謝承祐醫師,均已接受30學分身心障礙教育訓練。並領有ACLS高等心臟急救訓練合格證書。特殊需求牙科護士一名。
\end{outline}

\subsection{醫療設備}

\begin{outline}
\1 無障礙空間及設施
\1 個人防護,遵循感染控制標準流程
\1 牙科X光機、牙科診療椅
\1 家庭牙醫、兒童牙科及口腔顎面外科門診裝備,如洗牙機、強力抽吸設備與抽痰管、牙科治療器械、開口器、中醫雷射針灸
\1 生理監測器(血壓、血氧)
\1 急救車(心電圖、心臟電擊器),急救藥品(定期盤點控管)、急救設備(喉頭鏡、人工氣道氣管內管)、氧氣設備(壁式氧氣、節流裝置、氧氣面罩、潮氣瓶)
\end{outline}

\subsection{經營現況}

\begin{outline}
\1 每週開設2診(週四下午陳醫師照護成人、週六上午謝醫師照護18歲以下特需病患)
\1 執行口腔與牙齒治療,並對患者與其陪同家人進行口腔保健教育
\1 110年共計服務人次300人,個案管理追蹤人數60名
\end{outline}


\section{計畫內容}

\subsection{目的}
配合衛生福利部政策,推動「身心障礙牙科醫療服務網絡模式」雙向轉診,提供特殊需求者牙科醫療服務,以及口腔衛生保健雙向互動。


\subsection{計畫執行期程}
111年4月1日至12月31日

\subsection{獎助項目}
提供特殊需求者牙科醫療服務

\subsection{預估獎勵金額}
\begin{outline}
\1 計畫期間達成基本應辦理事項,請領獎勵費用計新臺幣(以下同)38萬元整
\1 每周開設特別門診3診,與8家醫療機構(其中3家需為未合作過之新增名單)建置轉診機制,另請領獎勵費用計5萬元整
\1 共計43萬元整
\end{outline}

\subsection{計畫期間辦理事項}
%依序撰寫各項預定辦理內容與進行步驟}

\begin{outline}
\1 每週開設特別門診3診(每診至少3小時)
    \2 週一上午祁力行醫師,到宅牙醫服務(特殊需求牙科負責護士陪同)
    \2 週三上午祁力行醫師(口腔顎面外科,備有雷射針灸)
    \2 週四下午陳培惠(家庭牙科)照護成人特需病患
    \2 週六上午謝承祐(兒童牙科)照護18歲以下特需病患
    
\1 進行口腔衛生教育,指導患者及其照顧家人
    \2 衛教人員由醫師及特殊需求牙科負責護士擔任
    \2 衛教內容包括心理建設、臉部減敏按摩、中醫穴道按摩、吞嚥練習、舌頭運動、牙齒清潔(小牙刷)及牙縫清潔(牙線棒、牙間刷)
    \2 衛教工作詳實紀錄於特殊需求者口腔衛教照護衛教紀錄表(如附件2-1)
    
\1 個案追蹤管理
    \2 由特殊需求牙科負責護士擔任
    \2 針對特殊需求者患者,於治療與衛教後次日,進行個案追蹤管理工作,詢問術後情況、口腔清潔狀況,安排並鼓勵每二個月定期回診/到宅
    \2 建立「病友」LINE群組,回傳照片、錄製影片加強衛教宣導
    \2 製作個案追蹤管理紀錄表(如附件2-2、2-3)

\1 身心障礙牙科醫療服務網絡模式
    \2 臺北市文山區已有14家身心障礙牙科醫療服務院所
 
%\1 特需服務院所(到院牙醫服務)
%\2 文山區計有萬芳醫院及13家診所,擬加強社區照護網絡,視案主病情須要,互相轉診,未來將由萬芳醫院及德威牙醫診所提供到宅牙醫服務
\end{outline}

臺北市文山區特須牙醫院所名單\\ % 1.0625
\begin{tabularx}{1.093\textwidth}{|c|p{3.2cm}|c|l|}
\hline
臺北&	童芯牙醫診所&	02-29366707 &	臺北市文山區久康街24巷7號1樓\\
\hline
臺北&	全家牙醫診所&	02-22346393 &	臺北市文山區木柵路2段143號\\
\hline
臺北 &	欣美牙醫診所 &	02-22343708 &	臺北市文山區木柵路3段10號(1、2樓)\\
\hline
臺北 &	永麗牙醫診所 &	02-29372227 &	臺北市文山區木新路3段111號(1、2樓)\\
\hline
臺北 &	天丞牙醫診所 &	02-29378380 &	臺北市文山區木新路3段325號\\
\hline
臺北 &	驊陽牙醫診所 &	02-86610188 &	臺北市文山區忠順街1段26巷32號\\
\hline
臺北 &	景華牙醫診所 &	02-29304167 &	臺北市文山區景華街122號1樓\\
\hline
臺北 &	德在牙醫診所 &	02-29338248 &	臺北市文山區景福街30號1樓\\
\hline
臺北 &	雅田牙醫診所 &	02-29352838 &	臺北市文山區景興路119號1樓\\
\hline
臺北 &	家樂牙醫診所 &	02-82301197 &	臺北市文山區萬安街33號\\
\hline
臺北 &	*德威牙醫診所 &	02-29328281 &	臺北市文山區興隆路1段72、74號、70巷1-1號2樓\\
\hline
臺北 &	臺北市立萬芳醫院-委託財團法人臺北醫學大學辦理
&	02-29307930 &	臺北市文山區興隆路3段111號\\
\hline
臺北 &	廣泉牙醫診所 &	02-29343572 &	臺北市文山區興隆路3段41號\\
\hline
臺北 &	德全牙醫診所 &	02-29341569 &	臺北市文山區羅斯福路6段248號1樓\\
\hline
\end{tabularx}\\
(*已提供到宅牙醫服務 \today)
   
\begin{outline}   
    \2 於計畫期間至少與8家院所建立並簽具合作契約書
    \2 主動與臺北市衛生局及臺北市牙醫師公會,會商建置區域內特殊需求者醫療轉診制度,增進合作醫院間交流活動
    \2 特殊需求者牙科醫療服務轉介單如附件2-4

\1 特需者口腔保健衛教講座
    \2 與身心障礙福利機構、特殊教育機構或特教班、聯合評估中心或發展遲緩療育機構、老人福利機構、長照機構或身心障礙團體等合作
    \2 舉辦6場講座,提供口腔保健衛教、口腔檢查,視該機構情況亦可能安排簡單診療服務(超音波洗牙、塗氟)
    \2 定期到萬芳醫院12樓護理之家,進行口腔保健衛教、口腔檢查,及簡單診療服務(拔牙、超音波洗牙、塗氟)
    \2 指導牙科PGY醫師進行社區服務
    
\1 特殊需求者牙科醫療培訓計畫
    \2 安排學員參加特殊需求者牙科醫療服務示範中心(如,臺大醫院、雙和醫院)舉辦之牙醫師培訓計畫
    \2 每年1名牙科部PGY醫師完成訓練,且領有證書
    \2 每週一上午到宅牙醫服務,由祁力行醫師、特殊需求牙科負責護士,帶領PGY醫師出診
    
\1 特殊需求者牙科醫療服務季報表
    \2 定期於111年4月15日、7月15日、10月15日及112年1月6日,以電子郵件方式回報
    \2 服務累計表(季報表)如附件3


\end{outline}

\subsection{回顧與展望}
\begin{outline}
\1 萬芳醫院牙科部,過去3年計畫執行情形及未來3年執行目標(附件1)
\1 計畫期間辦理事項摘要表(附件6)
\1 111年04月開辦到宅牙醫服務
\end{outline}
%衛生局指定辦理身心障礙者特別門診之資格證明文件(如公文,並得為影、複本)。

%服務人次:
%係特定障礙類別之特殊需求者,於本計畫獎助之醫院,接受具 10 學分以上身心障礙相關教育訓練之醫師所提供之牙科醫療服務人次總數(不限其接受診療時間是否為特別門診時段),惟不含該病人當次僅接受口腔健康狀況檢查、塗氟、特定或非特定局部治療、全身麻醉評估、開藥、抽血、X 光、拆線、全身麻醉回診、特殊牙周疾病基本控制處置、口腔衛教、查看傷口、拔牙回診...等處置者。

\subsection{醫療品質和成效}

\begin{outline}

%PDCA 
\1 特需者的醫療安全
    \2 針對每一位特需者的身體功能,評估潛在風險
    \2 結合現有病人安全及品質管理機制,並計畫(Plan)、執行(Do)、查核(Check)、行動(Act)
	\2 建立病人安全及品質管理作業流程
	\2 針對醫療事故潛在風險進行預防及管控
	\2 累積經驗、改善流程,辦理教育訓練,以傳承並持續精進
	
\1 特需者醫療品質及醫病共享決策
%https://sdm.patientsafety.mohw.gov.tw
    \2 醫病共享決策 (shared decision making, SDM),為有效率的解說工具,亦可提升專業形象
    \2 提供特需者/家人醫療選項(有醫學實證資訊),鼓勵表達自己的意願及期待
    \2 幫助照顧者積極參與病人安全、醫療決策,且學習照護知能
    \2 決策工具範例: 「根管治療後應選擇製作假牙(牙冠/牙套)或複合樹脂填補?」(衛福部公告版本 \url{https://sdm.patientsafety.mohw.gov.tw/AssistTool/AccessibilityForm?sn=24&tid=846DAB0DDB450C3D})
    
\1 靈性關懷
    \2 預立醫療照護諮商(Advance Care Planning, ACP) 後簽署「預立醫療決定」 (Advance Decision, AD)
    \2 意識清楚的特需者(特別是失能老人、重度以上心肺、吞嚥功能喪失之下,預先考量日後萬一處於特定臨床條件時,希望接受或拒絕之維持生命治療、人工營養及流體餵養或其他與醫療照護選項
    \2 四種特定臨床條件均有嚴格定義,包括生命末期病人、不可逆轉昏迷狀況、永久植物人狀態及極重度失智
    \2 幫助解決患者與家人的靈性困擾,以求靈性平安

\end{outline}

\subsection{預期效益}
特需病患面臨齲齒、牙周病細菌感染的惡化,因就醫之阻礙較多,照顧者也已疲於奔命,無暇再照顧患者的口腔衛生。
特殊需求者口腔醫學,著眼在量身打造的專業照顧,預期可減少患者口腔細菌造成的肺炎感染,亦能幫助糖尿病患者的血糖控制,進而預防患者身體的併發症、並減少照顧者的負擔。


\section{後續發展與推廣}
\subsection*{特殊需求者口腔醫學專科}
\addcontentsline{toc}{subsection}{特殊需求者口腔醫學專科}

萬芳醫院牙科部,預計持績推行特殊需求者到院及到宅服務,藉由對PGY醫師的帶領,傳承學識經驗,融合靈性關懷,建立到宅牙醫的標準安全模式,以為其他院所的參考。全人照護的心心相映,可將特殊需求者口腔醫學推向專科訓練的目標。
提升業務量後,增聘口腔衛生師一名,規劃獨立門診空間,並擴增專用牙科治療椅(三台)、有獨立的鎮靜麻醉設備與空間,輪 椅專用牙科X光機。
未來將申請加入臺灣特殊需求者口腔醫學會(Taiwan Association for Disabilities and Oral Health, TADOH)專科醫師訓練機構。
% http://www.tadoh.org.tw

%詳述計畫執行結束後之後續規劃,以達獎助後之永續責任

%訓練特色 以特殊需求者為照護對象:特需牙科的病人範疇包含身心障礙牙科、醫院牙科、老人牙科、長期口腔照護牙科、早期療育牙科五大區塊。從自閉症的孩童到阿茲海默症的老人等等都是我們看診的病人,

%訓練內容包含特殊需求者之牙體復形、根管治療、牙周治療、包含智齒與多生牙在內的恆牙拔牙、全口或局部假牙製作等全人醫療照護。此外,訓練醫師亦可以在實際看診、醫科科外訓練與病例討論會議中更深入了解各種疾病的相關醫療知識與緊急處理流程,將來看任何病人(不是只有特殊需求者)都能臨危不亂,處理得宜。
%醫科科外訓練:除了在特需牙科的訓練之外,也會安排醫科的科外訓練,到麻醉科、心臟內外科、兒童ICU病房、復健科、兒童心智科、基因醫學科等各科部進行學習,對各種系統性疾病有更近一步的認知,並運用於實際看診的病人上。

%麻醉手術實際操作:本科訓練醫師每週都會安排全身麻醉以及鎮靜麻醉的手術得以親自操作完成治療,並收集專科訓練案例。週五早上的麻醉科聯合晨會以及科外麻醉科的訓練,都可以讓受訓住院醫師增進手術相關的知識與技術,可以順利與麻醉醫師溝通,對於麻醉手術能更得心應手。
%笑氣麻醉操作與數位口掃:受訓醫師可以學習並操作笑氣鎮靜麻醉,得以應用於懼怕就診的病人上,也能帶領實習醫學生學習笑氣麻醉的操作技巧。數位口掃印模亦能運用於臨床,降低病人不適感,提高治療品質。


%\subsection{到宅牙醫標準模式}
%並可嘗試建立標準模式,提供其他地區標竿參考。
%全人照護靈性關懷



% my project main text
%% SWOT: Streghts, Weaknesses, Opportunities, Threats
% https://gist.github.com/ricardogarfe/4563453


\section{SWOT Matrix - \emph{(Strengths, Weaknesses, Opportunities, Threats)}}

% pentagon -> hexagon
\begin{tikzpicture}[
    hexagon/.style={%
        shape=regular polygon, regular polygon sides=6, minimum size=7.3cm, inner
        sep=-1mm, draw, fill=gray!10 %DarkSeaGreen!75!yellow
    }, font=\scriptsize\sffamily, thick
]

% \draw[help lines] (-16,-16) grid (16,16);
\filldraw[thin,gray,fill=gray!25] (-8,-8) rectangle (8,8);
\filldraw[thin,gray,fill=white] (-7.15,-7.15) rectangle (7.15,7.15);
\draw[thin,gray] (7.15,7.15)--(8,8) (-7.15,7.15)--(-8,8) (-7.15,-7.15)--(-8,-8)
(7.15,-7.15)--(8,-8);

% Strengths
% pentagon
\draw[thin, green] (-0.025,0.025)--(-7.05,0.025)--(-0.025,7.05)--cycle;

%\node[pentagon ,rotate=45] 
%\polygon[rotate=45]{6} 
\node[hexagon, rotate=45] at (-3.75,3.75) {
    \begin{varwidth}{\linewidth}
        \begin{itemize}[leftmargin=*,noitemsep]
            \item 社會責任與醫院評鑑
            %Technical and business expertise
            \item 守護文山新店深坑地區 %Domestic market orientation
            \item 偏鄉服務及特需中心 %初級
            %Stable management team
            \item 牙科部財務健全
            %Financial stability
            %\item Acquisition capabilities            
            \item 資深特需師資(四位) %七年30學分終身會員 %Economies of scale
            \item 複製北醫附醫五年到宅經驗 %Training programs
            %\item Loyalty and retention
        \end{itemize}
    \end{varwidth}
};
\draw (-2,2) node[rotate=45] {\large\textbf{Strengths}};

% Weaknesses
\draw[thin, brown!20!red] (0.025,0.025)--(7.05,0.025)--(0.025,7.05)--cycle;
\node[hexagon,rotate=-45] at (3.75,3.75) {
    \begin{varwidth}{\linewidth}
        \begin{itemize}[leftmargin=*,noitemsep]
            \item 初期經費與設備投資 %Centralized decisions
            \item  醫師、專責護士與人力規畫 %Accounts cross-selling
            \item 到宅牙醫行政文書
             %Marketing capabilities
            %\item 到宅醫療收入與時間成本 %Win on price image
            %\item 到宅交通費用 %BPO market
            %\item 特需案例來源
            %\item No differentiation
        \end{itemize}
    \end{varwidth}
};
\draw (2,2) node[rotate=-45] {\large\textbf{Weaknesses}};

% Opportunities
\draw[thin, green!50!blue] (-0.025,-0.025)--(-7.05,-0.025)--(-0.025,-7.05)--cycle;
%135
\node[hexagon,rotate=-45] at (-3.75,-3.75) {
    \begin{varwidth}{\linewidth}
        \begin{itemize}[leftmargin=*,noitemsep]
            \item  原PGY醫師(1人/月)支援連江縣立醫院 %Marketing push
            \item =>社區牙科訓練改為到宅牙醫服務 %Adding BPO capabilities
            \item 招募護士加入排班 %到宅醫師(二名排班) \item 到宅護士(二名排班)
            \item 牙科部祕書兼到宅牙醫行政
            \item 收取交通費/院方公務車支援 %Pricing structure
            \item 案例來源: 社區醫療部、出院準備服務 %Business process approach
            %\item %Annuity engagement
        \end{itemize}
    \end{varwidth}
};
\draw (-2,-2) node[rotate=-45] {\large\textbf{Opportunities}};

% Threats
\draw[thin, brown] (0.025,-0.025)--(7.05,-0.025)--(0.025,-7.05)--cycle;
\node[hexagon,rotate=45] at (3.75,-3.75) {
    \begin{varwidth}{\linewidth}
        \begin{itemize}[leftmargin=*,noitemsep]
            \item 到宅醫療收入不敷時間成本 %Profitability losses
            \item 收取到宅交通費形成門檻 %BPO market
            \item 特需案例來源
            \item 案主因病住院時,到宅服務暫停
            %\item High-risk deals
            %\item Image change inability
            %\item Degree of automation
        \end{itemize}
    \end{varwidth}
};
\draw (2,-2) node[rotate=45] {\large\textbf{Threats}};
%
\draw(0,-7.55) node {\Large EXTERNAL};
\draw(0,7.55) node {\Large INTERNAL};
\draw(-7.55,0) node[rotate=90] {\Large POSITIVE};
\draw(7.55,0) node[rotate=270] {\Large NEGATIVE};
\draw[green](-0.6,0.6) node {\Huge\textbf{S}}; 
\draw[brown!20!red](0.6,0.6) node {\Huge\textbf{W}};
\draw[green!50!blue](-0.6,-0.6) node {\Huge\textbf{O}};
\draw[brown](0.6,-0.6) node {\Huge\textbf{T}};
\end{tikzpicture}
  
  % treatment planning
% U:如何善用每個優勢? (How can we Use each Strength?)
%S:如何停止每個劣勢? (How can we Stop each Weakness?)
%E:如何成就每個機會? (How can we Exploit each Opportunity?)
%D:如何抵禦每個威脅? (How can we Defend against each Threat?)

臺北市立萬芳醫院牙科部特殊需求者牙科門診(特需中心)\\

\begin{outline}
% 零基預算 Zero-based budgeting system
% 成本中心
到宅牙醫經費概算:
\1 非專職人事(護士)月薪4.2萬元(每月到宅時數24/門診時數(160-24)=24/136=0.18),則到宅業務人事成本約為7412元; (依據勞基法「四週總工時不可超過160 小時」) %Required staffing to complete department functions; What staffing would be required at what experience levels & salaries
\1 設備(生理監測儀10萬元,攜帶型超音波洗牙機4萬元,攜帶型抽吸機0.5萬元) %Required equipment/software/hardware necessary to complete the function
\1 出診交通費(文山區200元,新店深坑400元,七堵暖暖700元) %Required overhead costs needed to complete the function
\1 耗材(每人次300元) %Required overhead costs needed to complete the function
\1 衛教宣導(禮品每次500元)
\1 教育訓練(講師鐘點費1000元)

\vspace{5mm}
預期效益及效益指標(key performance index, KPI): %Current marketing efforts and their efficiency
\1 每週出診數(每名醫師2次)、人次(首次家訪6,治療3)
\2 本計畫規範每位醫師執行特定需求者牙醫醫療服務及居家牙醫醫療服務(含評估訪視)合計每日達5人次以上,自第5人次起按6折支付,每日最多8人次為限,且每月以80人次為限;每位醫師支援每週不超過2日
\2 每人次健保收入(家訪1553元,治療5700元),交通費收入(200--700元)。
%,無轉診加成。
%(一)	預計提供高負荷家庭照顧者個案服務○○○人。
\2 每年預計提供家訪指導口腔照護及到宅治療及288人次/144人次
\2 KPI: 維持收案24名案主,每二個月穩定到宅治療一次
\2 每年健保收入小計1,268,064元
%(四)	預計拓展○○照顧資訊小站、○○照顧支持小站。
%(五)	預計辦理○○場教育訓練,受益人數為○○人次。
\1 預計辦理6場院內衛教宣導,擴及人數為60人次
\1 每年預計辦理長照家庭照顧者之口腔照護教育訓練2場,服務50人次

\end{outline}

2022年籌備期程(Gantt chart)\\
\begin{tabular}{|l|ccccccccc|}
\hline
 工作項目 & 四月 & 五月  & 六月 & 七月  & 八月  & 九月  & 十月  & 十一月  & 十二月  \\
\hline
 聘用護理人員    &  & &  & $\longrightarrow$ & $\longrightarrow$ & $\longrightarrow$ & $\longrightarrow$ & $\longrightarrow$ & $\longrightarrow$  \\
 設備採購    &  &  & $\longrightarrow$ & $\longrightarrow$ & &&&&  \\
 衛教及訓練    & $\longrightarrow$ & $\longrightarrow$ & $\longrightarrow$ & $\longrightarrow$  & $\longrightarrow$ & $\longrightarrow$ & $\longrightarrow$ & $\longrightarrow$ & $\longrightarrow$ \\
 到宅服務    &  & &  & $\longrightarrow$ & $\longrightarrow$ & $\longrightarrow$ & $\longrightarrow$ & $\longrightarrow$ & $\longrightarrow$  \\
 到院服務*  & $\longrightarrow$ $\longrightarrow$ & $\longrightarrow$ & $\longrightarrow$ & $\longrightarrow$  & $\longrightarrow$ & $\longrightarrow$ & $\longrightarrow$ & $\longrightarrow$ & $\longrightarrow$ \\ 
\hline
\end{tabular}\\
(*原已執行中的業務)

\clearpage
%\vspace{5mm}
區域內整體服務辦理情形 or
身心障礙牙科醫療服務網絡模式
特殊需求者牙科醫療服務轉介單(附件2-4)
%(盤點申請區域服務資源及說明單位組織量能、目前長照服務推動情形外,請加強敘明目前及未來辦理家庭照顧者服務情形): 社區資源開發整合
%https://www.nhi.gov.tw/BBS_Detail.aspx?n=73CEDFC921268679&sms=D6D5367550F18590&s=DC5108C6E3DD21D0

依據公告特需中心名單(2022年)
\begin{outline}
%\0 

\1 到宅牙醫服務
\2 全國136名醫師
\2 臺北市52名醫師
    \3 臺北醫學大學附設醫院4名醫師
    \3 文山區計有德威牙醫診所2名醫師
    %陳義聰、蕭雅純二名醫師
\2 新北市: 雙和醫院8名醫師

\1 特需服務院所(到院牙醫服務)
\2 文山區計有萬芳醫院及13家診所,擬加強社區照護網絡,視案主病情須要,互相轉診,未來將由萬芳醫院及德威牙醫診所提供到宅牙醫服務
\end{outline}

臺北市文山區特須牙醫院所名單\\
\begin{tabularx}{1.0625\textwidth}{|c|p{3.2cm}|c|l|}
\hline
臺北&	童芯牙醫診所&	02-29366707 &	臺北市文山區久康街24巷7號1樓\\
\hline
臺北&	全家牙醫診所&	02-22346393 &	臺北市文山區木柵路2段143號\\
\hline
臺北 &	欣美牙醫診所 &	02-22343708 &	臺北市文山區木柵路3段10號(1、2樓)\\
\hline
臺北 &	永麗牙醫診所 &	02-29372227 &	臺北市文山區木新路3段111號(1、2樓)\\
\hline
臺北 &	天丞牙醫診所 &	02-29378380 &	臺北市文山區木新路3段325號\\
\hline
臺北 &	驊陽牙醫診所 &	02-86610188 &	臺北市文山區忠順街1段26巷32號\\
\hline
臺北 &	景華牙醫診所 &	02-29304167 &	臺北市文山區景華街122號1樓\\
\hline
臺北 &	德在牙醫診所 &	02-29338248 &	臺北市文山區景福街30號1樓\\
\hline
臺北 &	雅田牙醫診所 &	02-29352838 &	臺北市文山區景興路119號1樓\\
\hline
臺北 &	家樂牙醫診所 &	02-82301197 &	臺北市文山區萬安街33號\\
\hline
臺北 &	*德威牙醫診所 &	02-29328281 &	臺北市文山區興隆路1段72、74號、70巷1-1號2樓\\
\hline
臺北 &	臺北市立萬芳醫院-委託財團法人臺北醫學大學辦理
&	02-29307930 &	臺北市文山區興隆路3段111號\\
\hline
臺北 &	廣泉牙醫診所 &	02-29343572 &	臺北市文山區興隆路3段41號\\
\hline
臺北 &	德全牙醫診所 &	02-29341569 &	臺北市文山區羅斯福路6段248號1樓\\
\hline
\end{tabularx}\\
(*已提供到宅牙醫服務 \today)

% https://ceriniandassociates.com/news-feed/2020/07/13/zero-based-budgeting/
%One tool which can be helpful to practices of all sizes can be to periodically implement a “zero-based” budgeting approach. Zero-based budgeting was first developed and implemented in the 1970s and then driven to extremes by investment firms such as 3G Capital in the 2000s as a tool to cut costs wherever possible and to extremes, such as focusing on such minute details as the number of pages printed and photocopies made by employees. As a result, zero-based budgeting has a reputation as an “austerity” measure and a tool used to cut costs to the bone. This zealous approach to the ideals may have its place in many organizations, however, the original ideals and principles of the system should be part of any organization’s planning toolbox from time-to-time.

%In a “zero-based” budgeting system, departments should look periodically at each year and budget as if starting from a zero-dollar budget allocation, rather than just what was spent in the past. The real point of the exercise is to take a top-down approach and determine if all spend in any given department is required to fulfill the functions of this department. The additional scrutiny can be used to uncover potential inefficiencies, over or understaffing in departments, discover potential synergies between departments, and empower department heads to perform an overall review of their operations.

%In implanting a zero-based budgeting system, a healthcare organization should take the following steps:
%1.) Start a baseline zero for all departments; prior-year spending does not matter.
%2.) Evaluate every cost area within the department (in conjunction with the department head). This evaluation should include:
%a.) Required staffing to complete department functions
%b.) Required equipment/software/hardware necessary to complete the function
%c.) Required overhead costs needed to complete the function

%When evaluating, consider how one would start *** a brand-new department from scratch. 
%What staffing would be required at what experience levels & salaries; what costs are required to perform the necessary operations? 
%The critical eye here is necessary and often best done in a collaborative effort with both an insider (someone in the department) and an outsider (someone outside the department). 外部委員

%3.) Justify the spending in these above areas and try to identify cost savings. Some examples may include:
%a.) Looking at the cost of outsourcing billing vs. proving billing services in-house
%b.) Staffing levels of reception/administrative staff
%c.) Current marketing efforts and their efficiency


%\bibliographystyle{plainnat}
%\bibliography{ref}

\newpage
\appendix
%\input{content/appendix}
\includepdf[pages=-]{111年牙醫特殊醫療服務計畫 (ICF).pdf}
\includepdf[pages=-]{111年度申請補助作業規定_一般醫院 (summary6).pdf} % (附件6)
%\includegraphics[width=0.9\textwidth]{111年牙醫特殊醫療服務計畫 (附件3).png}
\clearpage
%\includepdf[pages=-]{111年牙醫特殊醫療服務計畫 (附件7).pdf}
%\includegraphics[width=0.9\textwidth]{111年牙醫特殊醫療服務計畫 (附件7).png}
%\includepdf[pages=-]{111年牙醫特殊醫療服務計畫 (附件12).pdf} % 附件12 診紀錄
%\includepdf[pages=-]{A040170081040500-1090331-7000-014.pdf} % 附件14
%\includepdf[pages=-]{111年牙醫特殊醫療服務計畫 (DOH_附件).pdf}

%% Please add the following required packages to your document preamble:
% \usepackage{graphicx}

【附件19】

\begin{table}[H]
%\caption{}
%\label{tab:DOH_acls}
\centering
\resizebox{\textwidth}{!}{%
\begin{tabular}{|l|l|l|l|}
\hline
分類       & 名稱                           & 功能                                                                   & 備註      \\ \hline
BLS &
  Vital sign monitor &
  \begin{tabular}[c]{@{}l@{}}NIBP blood pressure (sphygmomanometer),\\ pulse oximeter (oxygen saturation, SpO2), pulse rate, BPM\end{tabular} &
  HP/Phillip \\ \hline
       & Glucose Meter                & blood sugar from finger tip                                          &         \\ \hline
       & Stethoscope                  & 心臟科聽診器                                                               &         \\ \hline
       & Oxygen supply                & ambu bag, face mask 6L/min; nasal prong (with cannula)               & O2 tank \\ \hline
       & Airway                       & nasopharyngeal airway (nasal trumpet, size #7 #8)                    &         \\ \hline
 &
  Airway (advanced) &
  \begin{tabular}[c]{@{}l@{}}laryngeal mask airway (LMA), oral endotracheal tube (French #6, #7)\\ laryngoscope (插管用喉鏡 curved blade)/Magi forceps + Lidocaine jelly\end{tabular} &
   \\ \hline
Drug   & Allergy with broncospasm     & 肌肉內注射 0.2-0.5mg epinephrine (1 mg/mL)                                &         \\ \hline
       & Status epilepticus (> 5 min) & diazepam 肌肉或靜脈注射(靜脈注射較佳),初劑量 5-10mg                                  &         \\ \hline
Dental & 攜帶式超音波洗牙機                    & NSK Varios 970 LUX (water supply, fibroptic cable)                   &         \\ \hline
       & 攜帶式真空吸引機(suction machine)    & 抽痰管, suction tip with saliva ejector, 治療巾, 彎盆                        &         \\ \hline
       & PPE                          & 口罩surgical mask 隔離衣gown 護目鏡goggle                                    & x3      \\ \hline
       & Fluoridation                 & 氟膠 Sodium Fluoride gel 5\%(Clinpro[3M] vanish with brush), 2x2 gauze &         \\ \hline
 &
  衛教 &
  \begin{tabular}[c]{@{}l@{}}張口棒(健康牌)、木質壓舌板、\\ 兒童牙刷(soft brush, flat handle)、牙間刷、3M牙線棒\end{tabular} &
   \\ \hline
       & 感控                           & 75\% alcohol swab、乾洗手液,檢診手套,紅色感染用塑膠袋                                 &         \\ \hline
Others &                              & 手電筒, 電源延長線/adaptor                                                   &         \\ \hline
\end{tabular}%
}

\end{table} % 附件19

\end{document}
