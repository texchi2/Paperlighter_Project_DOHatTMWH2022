%{
%一、計畫書封面:至少包含計畫名稱(包含計畫執行地區)、計畫執行單位、計 畫執行期間。
%二、書寫格式:以 word 建檔,A4 版面,由左而右,由上而下,標楷體 14 號字 型,橫式書寫,填寫綜合資料表,另計畫書編有目錄頁碼。

%三、計畫本文至少應包括: (一)緣起/前言:請敘述申請本計畫產生之背景。 
%(二)現況分析:請敘述現實施地區所呈現問題。 
%(三)計畫人力配置:組織架構、現況、醫事人力、醫療設備、經營現況;另詳述醫事人力(專任或兼任醫師、執業登錄)及參與獎助服務醫師名冊 並檢附受有 10 學分以上身心障礙相關教育訓練之證明文件(僅需列出指 定課程 10 個學分數,並完整確實標註清楚)。

%(四)計畫內容: 
%1、目的。
%2、計畫執行期程、申請之獎助項目及預估獎勵金額。
%3、依序撰寫各項預定辦理內容與進行步驟。
%4、過去 3 年計畫執行情形及未來 3 年執行目標、摘要表(如附件 1、6)、衛生局指定辦理身心障礙者特別門診之資格證明文件(如公文,並得為影、複本)。
%5、品質和成效。 
%6、預期效益。

%(五)後續發展或推廣:詳述計畫執行結束後之後續規劃,以達獎助後之永續責任,並可嘗試建立標準模式,提供其他地區標竿參考。

%}
% 口腔醫療與保健推廣及保健推廣計畫書
% 僅第一項: 補助醫院提供特殊需求者牙科醫療服務
\section{緣起}
臺北市立萬芳醫院牙科部設置特殊需求者牙科門診(以下稱特需中心),每週門診2診次,專門服務身心障礙者,牙科櫃台附有身心障礙服務窗口,設置志工,與負責身心障礙門診之牙科輔助人員共同幫忙引導、溝通、協助患者就醫。候診區及初診區均有輪椅無障礙空間。
%先申請特需中心,然後牙科部專任醫師才具有到宅牙醫服務的資格,當然也需要身障照護的學分。到宅小組需要一名醫師、一名護理師,攜帶必要設備,每一名案主收案都事先申請核可
特殊需求者的口腔照護,因日常生活打理由照顧者代勞,往往無法顧及基本的口腔衛生照護,行動不便、甚至長期在宅臥床的患者更面臨齲齒、牙周病細菌感染的重大危機,因其就醫之阻礙甚多,四、五層樓之公寓往往沒有電梯。口腔細菌感染有機會吸入造成肺炎感染,增加患者身體的負擔。
萬芳醫院牙科部特需中心,配合長照2.0政策,除服務原有的到院病患,更自111年3月起進一步籌備居家牙醫醫療服務(以下稱到宅牙醫)。

\section{現況分析}
身心障礙者特別門診,每週開設2診(週四下午陳培惠醫師、週六上午謝承祐醫師)。家庭牙醫科陳醫師,及兒童牙科謝醫師,均已接受30學分身心障礙教育訓練。除執行適當的口腔與牙齒治療,並對患者與其陪同家人進行口腔保健教育。110年共計服務人次300人,個案管理追蹤人數60名。
	
%「全民健康保險會」協定年度醫療給付費用總額事項」辦理
% https://www.nhi.gov.tw/BBS_Detail.aspx?n=73CEDFC921268679&sms=D6D5367550F18590&s=DC5108C6E3DD21D0

本院牙科部依據「111年全民健康保險牙醫門診總額特殊醫療服務計畫」,辦理「特殊需求者到宅牙醫計畫」,旨在擴大特定身心障礙者口腔醫療與保健推廣。預計每年可穩定服務(每名患者每2個月一次)20名以上的居家醫療需求者。
%(以下稱本計畫)。

\section{計畫人力配置}

\subsection{組織架構}
\begin{outline}

%\1 組織架構
\1 現況: 萬芳醫院牙科部,「特殊需求者口腔醫學科」籌備處,規畫整合原有特需門診,加上到宅牙醫服務,形成完整的特需服務與衛生教育推廣網絡
\1 醫事人力: %詳述醫事人力(專任或兼任醫師、執業登錄)及參與獎助服務醫師名冊並檢附受有 10 學分以上身心障礙相關教育訓練之證明文件(僅需列出指定課程10個學分數,並完整確實標註清楚)。
本院專任醫師三名,口腔顎面外科主任祁力行醫師、
家庭牙醫科主任陳培惠醫師、
兒童牙科主任謝承祐醫師,均已接受30學分身心障礙教育訓練。並領有ACLS高等心臟急救訓練合格證書。特殊需求牙科護士一名。
\end{outline}

\subsection{醫療設備}

\begin{outline}
\1 無障礙空間及設施
\1 個人防護,遵循感染控制標準流程
\1 牙科X光機、牙科診療椅
\1 家庭牙醫、兒童牙科及口腔顎面外科門診裝備,如洗牙機、強力抽吸設備與抽痰管、牙科治療器械、開口器、中醫雷射針灸
\1 生理監測器(血壓、血氧)
\1 急救車(心電圖、心臟電擊器),急救藥品(定期盤點控管)、急救設備(喉頭鏡、人工氣道氣管內管)、氧氣設備(壁式氧氣、節流裝置、氧氣面罩、潮氣瓶)
\end{outline}

\subsection{經營現況}

\begin{outline}
\1 每週開設2診(週四下午陳醫師照護成人、週六上午謝醫師照護18歲以下特需病患)
\1 執行口腔與牙齒治療,並對患者與其陪同家人進行口腔保健教育
\1 110年共計服務人次300人,個案管理追蹤人數60名
\end{outline}


\section{計畫內容}

\subsection{目的}
配合衛生福利部政策,推動「身心障礙牙科醫療服務網絡模式」雙向轉診,提供特殊需求者牙科醫療服務,以及口腔衛生保健雙向互動。


\subsection{計畫執行期程}
111年4月1日至12月31日

\subsection{獎助項目}
提供特殊需求者牙科醫療服務

\subsection{預估獎勵金額}
\begin{outline}
\1 計畫期間達成基本應辦理事項,請領獎勵費用計新臺幣(以下同)38萬元整
\1 每周開設特別門診3診,與8家醫療機構(其中3家需為未合作過之新增名單)建置轉診機制,另請領獎勵費用計5萬元整
\1 共計43萬元整
\end{outline}

\subsection{計畫期間辦理事項}
%依序撰寫各項預定辦理內容與進行步驟}

\begin{outline}
\1 每週開設特別門診3診(每診至少3小時)
    \2 週一上午祁力行醫師,到宅牙醫服務(特殊需求牙科負責護士陪同)
    \2 週三上午祁力行醫師(口腔顎面外科,備有雷射針灸)
    \2 週四下午陳培惠(家庭牙科)照護成人特需病患
    \2 週六上午謝承祐(兒童牙科)照護18歲以下特需病患
    
\1 進行口腔衛生教育,指導患者及其照顧家人
    \2 衛教人員由醫師及特殊需求牙科負責護士擔任
    \2 衛教內容包括心理建設、臉部減敏按摩、中醫穴道按摩、吞嚥練習、舌頭運動、牙齒清潔(小牙刷)及牙縫清潔(牙線棒、牙間刷)
    \2 衛教工作詳實紀錄於特殊需求者口腔衛教照護衛教紀錄表(如附件2-1)
    
\1 個案追蹤管理
    \2 由特殊需求牙科負責護士擔任
    \2 針對特殊需求者患者,於治療與衛教後次日,進行個案追蹤管理工作,詢問術後情況、口腔清潔狀況,安排並鼓勵每二個月定期回診/到宅
    \2 建立「病友」LINE群組,回傳照片、錄製影片加強衛教宣導
    \2 製作個案追蹤管理紀錄表(如附件2-2、2-3)

\1 身心障礙牙科醫療服務網絡模式
    \2 臺北市文山區已有14家身心障礙牙科醫療服務院所
    \2 於計畫期間至少與8家院所建立並簽具合作契約書
    \2 主動與臺北市衛生局及臺北市牙醫師公會,會商建置區域內特殊需求者醫療轉診制度,增進合作醫院間交流活動
    \2 特殊需求者牙科醫療服務轉介單如附件2-4

\1 特需者口腔保健衛教講座
    \2 與身心障礙福利機構、特殊教育機構或特教班、聯合評估中心或發展遲緩療育機構、老人福利機構、長照機構或身心障礙團體等合作
    \2 舉辦6場講座,提供口腔保健衛教、口腔檢查,視該機構情況亦可能安排簡單診療服務(超音波洗牙、塗氟)
    \2 定期到萬芳醫院12樓護理之家,進行口腔保健衛教、口腔檢查,及簡單診療服務(拔牙、超音波洗牙、塗氟)
    \2 指導牙科PGY醫師進行社區服務
    
\1 特殊需求者牙科醫療培訓計畫
    \2 安排學員參加特殊需求者牙科醫療服務示範中心(如,臺大醫院、雙和醫院)舉辦之牙醫師培訓計畫
    \2 每年1名牙科部PGY醫師完成訓練,且領有證書
    \2 每週一上午到宅牙醫服務,由祁力行醫師、特殊需求牙科負責護士,帶領PGY醫師出診
    
\1 特殊需求者牙科醫療服務季報表
    \2 定期於111年4月15日、7月15日、10月15日及112年1月6日,以電子郵件方式回報
    \2 服務累計表(季報表)如附件3


\end{outline}

\subsection{回顧與展望}
\begin{outline}
\1 萬芳醫院牙科部,過去3年計畫執行情形及未來3年執行目標(附件1)
\1 計畫期間辦理事項摘要表(附件16)
\1 111年04月開辦到宅牙醫服務
\end{outline}
%衛生局指定辦理身心障礙者特別門診之資格證明文件(如公文,並得為影、複本)。

%服務人次:
%係特定障礙類別之特殊需求者,於本計畫獎助之醫院,接受具 10 學分以上身心障礙相關教育訓練之醫師所提供之牙科醫療服務人次總數(不限其接受診療時間是否為特別門診時段),惟不含該病人當次僅接受口腔健康狀況檢查、塗氟、特定或非特定局部治療、全身麻醉評估、開藥、抽血、X 光、拆線、全身麻醉回診、特殊牙周疾病基本控制處置、口腔衛教、查看傷口、拔牙回診...等處置者。

\subsection{品質和成效}

\subsection{預期效益}


\section{後續發展或推廣}
\subsection{特殊需求者口腔醫學專科訓練}
臺灣特殊需求者口腔醫學會專科醫師
詳述計畫執行結束後之後續規劃,以達獎助後之永續責任

【110學年度 台大醫院 特殊需求者牙科醫療服務示範中心第三年住院醫師招募】
1 訓練特色
1.1 以特殊需求者為照護對象:特需牙科的病人範疇包含身心障礙牙科、醫院牙科、老人牙科、長期口腔照護牙科、早期療育牙科五大區塊。從自閉症的孩童到阿茲海默症的老人等等都是我們看診的病人,範圍非常廣泛,長期的回診追蹤的老病人也非常多。訓練內容包含特殊需求者之牙體復形、根管治療、牙周治療、包含智齒與多生牙在內的恆牙拔牙、全口或局部假牙製作等全人醫療照護。此外,訓練醫師亦可以在實際看診、醫科科外訓練與病例討論會議中更深入了解各種疾病的相關醫療知識與緊急處理流程,將來看任何病人(不是只有特殊需求者)都能臨危不亂,處理得宜。
1.2 醫科科外訓練:除了在特需牙科的訓練之外,也會安排醫科的科外訓練,到麻醉科、心臟內外科、兒童ICU病房、復健科、兒童心智科、基因醫學科等各科部進行學習,對各種系統性疾病有更近一步的認知,並運用於實際看診的病人上。
1.3 國外進修:本科之訓練醫師可以申請至美國加州大學洛杉磯分校(UCLA)牙醫學院進行為期兩個月的進修,以精進醫院牙科(Hospital dentitstry)為主的學術交流。
1.4 麻醉手術實際操作:本科訓練醫師每週都會安排全身麻醉以及鎮靜麻醉的手術得以親自操作完成治療,並收集專科訓練案例。週五早上的麻醉科聯合晨會以及科外麻醉科的訓練,都可以讓受訓住院醫師增進手術相關的知識與技術,可以順利與麻醉醫師溝通,對於麻醉手術能更得心應手。
1.5 笑氣麻醉操作與數位口掃:受訓醫師可以學習並操作笑氣鎮靜麻醉,得以應用於懼怕就診的病人上,也能帶領實習醫學生學習笑氣麻醉的操作技巧。數位口掃印模亦能運用於臨床,降低病人不適感,提高治療品質。
1.6 可依興趣選擇台大牙科其他專科,進行第二專長訓練。
1.7 交通好便利、生活真方便:北車、捷運站、機場捷運線,交通便利,餐廳多元。
1.8 完訓後可以申請台大特需牙科兼任主治醫師職位。
2 臨床時數與Loading
2.1 第三年起之住院醫師不用值班,沒有夜診,六日也沒有上班。例假日依照醫院標準放假。
2.2 平日早午診上班,算入勞基法上班時數。
2.3 臨床人力充足,助理與護理師可以提供四手到六手操作。
3 其他福利
3.1 有年終獎金、生日、三節獎金。
3.2 提供台大醫院宿舍,雙人房(1500元)及四人房(1000元)。
3.3 年假依照勞基法。
4 未來發展
4.1 訓練特殊需求者及一般患者全人照護醫療
4.2 醫科相關科外訓練
4.3 全身麻醉與鎮靜麻醉牙科治療訓練
4.4 完訓後可以申請台大特需牙科兼任主治醫師職位
4.5不會被強迫只能看特需病人,而是超級GP,兼具口外,兒牙能力


\subsection{到宅牙醫標準模式}
並可嘗試建立標準模式,提供其他地區標竿參考。


