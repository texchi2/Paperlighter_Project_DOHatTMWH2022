%{
%一、計畫書封面:至少包含計畫名稱(包含計畫執行地區)、計畫執行單位、計 畫執行期間。
%二、書寫格式:以 word 建檔,A4 版面,由左而右,由上而下,標楷體 14 號字 型,橫式書寫,填寫綜合資料表,另計畫書編有目錄頁碼。

%三、計畫本文至少應包括: (一)緣起/前言:請敘述申請本計畫產生之背景。 
%(二)現況分析:請敘述現實施地區所呈現問題。 
%(三)計畫人力配置:組織架構、現況、醫事人力、醫療設備、經營現況;另詳述醫事人力(專任或兼任醫師、執業登錄)及參與獎助服務醫師名冊 並檢附受有 10 學分以上身心障礙相關教育訓練之證明文件(僅需列出指 定課程 10 個學分數,並完整確實標註清楚)。

(%四)計畫內容: 
%1、目的。
%2、計畫執行期程、申請之獎助項目及預估獎勵金額。
%3、依序撰寫各項預定辦理內容與進行步驟。
%4、過去 3 年計畫執行情形及未來 3 年執行目標、摘要表(如附件 1、6)、衛生局指定辦理身心障礙者特別門診之資格證明文件(如公文,並得為影、複本)。
%5、品質和成效。 
%6、預期效益。

%(五)後續發展或推廣:詳述計畫執行結束後之後續規劃,以達獎助後之永續責任,並可嘗試建立標準模式,提供其他地區標竿參考。

%}

\section*{緣起/前言.}
臺北市立萬芳醫院牙科部設置特殊需求者牙科門診(以下稱特需中心),每週門診2診次,專門服務身心障礙者,牙科櫃台附有身心障礙服務窗口,設置志工,與負責身心障礙門診之牙科輔助人員共同幫忙引導、溝通、協助患者就醫。候診區及初診區均有輪椅無障礙空間。
%先申請特需中心,然後牙科部專任醫師才具有到宅牙醫服務的資格,當然也需要身障照護的學分。到宅小組需要一名醫師、一名護理師,攜帶必要設備,每一名案主收案都事先申請核可
特殊需求者的口腔照護,因日常生活打理由照顧者代勞,往往無法顧及基本的口腔衛生照護,在宅臥床患者更面臨齲齒、牙周病細菌感染的重大危機,因其就醫之阻礙甚多,四、五層樓之公寓往往沒有電梯。
特需中心配合長照2.0政策,規畫居家牙醫醫療服務。

\section*{現況分析.}
%「全民健康保險會」協定年度醫療給付費用總額事項」辦理
% https://www.nhi.gov.tw/BBS_Detail.aspx?n=73CEDFC921268679&sms=D6D5367550F18590&s=DC5108C6E3DD21D0
「111年全民健康保險牙醫門診總額特殊醫療服務計畫」辦理「特殊需求者口腔醫療與保健推廣計畫」(以下稱本計畫)。

\section*{計畫人力配置.} 旨在提升特定身心障礙者牙醫醫療服務品質,並加強口腔保健。

\section*{計畫內容.}

\subsection*{目的}
\subsection*{計畫執行期程、申請之獎助項目及預估獎勵金額}
\subsection*{依序撰寫各項預定辦理內容與進行步驟}
\subsection*{過去 3 年計畫執行情形及未來 3 年執行目標}
摘要表(如附件1、6)、衛生局指定辦理身心障礙者特別門診之資格證明文件(如公文,並得為影、複本)。
\subsection*{品質和成效}
\subsection*{預期效益}


\section{後續發展或推廣}

\subsection{x醫院牙醫醫療服務(以下稱到院牙醫服務)}
\label{dent}

\subsubsection{x特需中心}

\begin{outline}

%\1 衛生福利部「特殊需求者牙科醫療服務示範中心獎勵計畫」之醫院 
%or 「特殊需求者牙科醫療服務示範中心獎助計畫」及「特殊需求者牙科醫療服務獎助計畫」申請補助作業
\1 初級照護院所:
台北市立萬芳醫院-委託
財團法人私立台北醫學大
學辦理(醫事機構代號1301200010),審核已通過1110101至1111231。
% https://cda.org.tw/cda/public_medical_institution_type_b_search_result.jsp
%(1)申請書格式如【附件3】。 %(2)身心障礙教育訓練之學分證明影本。
%(3)牙醫師證書正反面影本一份。
\2 醫院資格
%    \3 可施行鎮靜麻醉之醫療院所及提供完備醫療之醫護人員。
    \3 設備需求:牙科門診應有急救設備、氧氣設備。
    %、麻醉機、心電 圖裝置(Monitor,包括血壓、脈搏、呼吸數之監測、血氧濃度pulse oximeter)、無障礙空間及設施(詳【附表】)。
%    \3 需1位以上具有從事相關工作經驗之醫師。
\2 醫師資格
    \3 %2位以上具有從事相關工作經驗之醫師,負責醫師自執業執照取得後滿5年以上之臨床經驗,其他醫師
    自執業執照取得後滿1年以上之臨床經驗之醫師。
    \3 經身心障礙口腔醫療教育課程(詳\ref{certificate})訓練合格。

\1 進階照護院所 %進階照護院所:口腔醫療與保健推廣計畫書
\2 醫院資格
    \3 可施行鎮靜麻醉之醫療院所及提供完備醫療之醫護人員。
    \3 設備需求:牙科門診應有急救設備、氧氣設備、麻醉機、心電 圖裝置(Monitor,包括血壓、脈搏、呼吸數之監測、血氧濃度pulse oximeter)、無障礙空間及設施(詳【附表】)。
%    \3 需2 位以上具有從事相關工作經驗之醫師。
\2 醫師資格
    \3 2位以上具有從事相關工作經驗之醫師,負責醫師自執業執照取得後滿5年以上之臨床經驗,其他醫師自執業執照取得後滿1年以上之臨床經驗。
    \3 經身心障礙口腔醫療教育課程(詳\ref{certificate})訓練合格。
\end{outline}

\subsubsection{x申請方式}