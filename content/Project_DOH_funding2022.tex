%{
%一、計畫書封面:至少包含計畫名稱(包含計畫執行地區)、計畫執行單位、計 畫執行期間。
%二、書寫格式:以 word 建檔,A4 版面,由左而右,由上而下,標楷體 14 號字 型,橫式書寫,填寫綜合資料表,另計畫書編有目錄頁碼。

%三、計畫本文至少應包括: (一)緣起/前言:請敘述申請本計畫產生之背景。 
%(二)現況分析:請敘述現實施地區所呈現問題。 
%(三)計畫人力配置:組織架構、現況、醫事人力、醫療設備、經營現況;另詳述醫事人力(專任或兼任醫師、執業登錄)及參與獎助服務醫師名冊 並檢附受有 10 學分以上身心障礙相關教育訓練之證明文件(僅需列出指 定課程 10 個學分數,並完整確實標註清楚)。

%(四)計畫內容: 
%1、目的。
%2、計畫執行期程、申請之獎助項目及預估獎勵金額。
%3、依序撰寫各項預定辦理內容與進行步驟。
%4、過去 3 年計畫執行情形及未來 3 年執行目標、摘要表(如附件 1、6)、衛生局指定辦理身心障礙者特別門診之資格證明文件(如公文,並得為影、複本)。
%5、品質和成效。 
%6、預期效益。

%(五)後續發展或推廣:詳述計畫執行結束後之後續規劃,以達獎助後之永續責任,並可嘗試建立標準模式,提供其他地區標竿參考。

%}
% 口腔醫療與保健推廣及保健推廣計畫書
% 僅第一項: 補助醫院提供特殊需求者牙科醫療服務
\section*{緣起.}
臺北市立萬芳醫院牙科部設置特殊需求者牙科門診(以下稱特需中心),每週門診2診次,專門服務身心障礙者,牙科櫃台附有身心障礙服務窗口,設置志工,與負責身心障礙門診之牙科輔助人員共同幫忙引導、溝通、協助患者就醫。候診區及初診區均有輪椅無障礙空間。
%先申請特需中心,然後牙科部專任醫師才具有到宅牙醫服務的資格,當然也需要身障照護的學分。到宅小組需要一名醫師、一名護理師,攜帶必要設備,每一名案主收案都事先申請核可
特殊需求者的口腔照護,因日常生活打理由照顧者代勞,往往無法顧及基本的口腔衛生照護,在宅臥床患者更面臨齲齒、牙周病細菌感染的重大危機,因其就醫之阻礙甚多,四、五層樓之公寓往往沒有電梯。
萬芳醫院牙科部特需中心,配合長照2.0政策,除服務原有的到院病患,更自111年3月起進一步籌備居家牙醫醫療服務(以下稱到宅牙醫)。

\section*{現況分析.}
身心障礙者特別門診,每週開設2診(週四下午陳培惠醫師、週六上午謝承祐醫師)。家庭牙醫科陳醫師,及兒童牙科謝醫師,均已接受30學分身心障礙教育訓練。除執行適當的口腔與牙齒治療,並對患者與其陪同家人進行口腔保健教育。110年共計服務人次300人,個案管理追蹤人數60名。
	
%「全民健康保險會」協定年度醫療給付費用總額事項」辦理
% https://www.nhi.gov.tw/BBS_Detail.aspx?n=73CEDFC921268679&sms=D6D5367550F18590&s=DC5108C6E3DD21D0

本院牙科部依據「111年全民健康保險牙醫門診總額特殊醫療服務計畫」,辦理「特殊需求者到宅牙醫計畫」,旨在擴大特定身心障礙者口腔醫療與保健推廣。預計每年可穩定服務(每名患者每2個月一次)20名以上的居家醫療需求者。
%(以下稱本計畫)。

\section*{計畫人力配置.}
\begin{outline}

\1 組織架構
\2 現況: 萬芳醫院牙科部,「特殊需求者口腔醫學科」籌備處,規畫整合原有特需門診,加上到宅牙醫服務,形成完整的特需服務與衛生教育推廣網絡
\2 醫事人力: %詳述醫事人力(專任或兼任醫師、執業登錄)及參與獎助服務醫師名冊並檢附受有 10 學分以上身心障礙相關教育訓練之證明文件(僅需列出指定課程10個學分數,並完整確實標註清楚)。
本院專任醫師三名,口腔顎面外科主任祁力行醫師、
家庭牙醫科主任陳培惠醫師、
兒童牙科主任謝承祐醫師,均已接受30學分身心障礙教育訓練。並領有ACLS高等心臟急救訓練合格證書

\1 醫療設備: 
\2 無障礙空間及設施
\2 個人防護,遵循感染控制標準流程
\2 牙科X光機、牙科診療椅
\2 家庭牙醫、兒童牙科及口腔顎面外科門診裝備,如洗牙機、強力抽吸設備與抽痰管、牙科治療器械、開口器
\2 生理監測器(血壓、血氧)
\2 急救車(心電圖、心臟電擊器),急救藥品(定期盤點控管)、急救設備(喉頭鏡、人工氣道氣管內管)、氧氣設備(壁式氧氣、節流裝置、氧氣面罩、潮氣瓶)

\1 經營現況
\2 每週開設2診(週四下午陳醫師照護成人、週六上午謝醫師照護18歲以下特需病患)
\2 執行口腔與牙齒治療,並對患者與其陪同家人進行口腔保健教育
\2 110年共計服務人次300人,個案管理追蹤人數60名
\end{outline}


\section*{計畫內容.}

\subsection{目的}

\subsection{計畫執行期程、申請之獎助項目及預估獎勵金額}

\subsection{計畫期間應辦理事項}
%依序撰寫各項預定辦理內容與進行步驟}

\begin{outline}
\1 每週開設特別門診 2 診(每診至少 3 小時)
\1 應對結束治療後之病人,進行適當之衛教工作並填寫紀錄(本部特殊需求者口腔衛教照護衛教紀錄表如附件 2-1)。(請於申請文件敘明衛教人員資格及衛教工作內容)
\1 每月至少對未接受全身麻醉或靜脈鎮靜病人之 50\%,進行個案追蹤管理工作並填寫紀錄(本部麻醉個案追蹤與副作用或併發症記錄單及個案追蹤管理表如附件 2-2、2-3)。
\1 至少與 7 家鄰近醫療院所合作並簽具合作契約書(其中 2 家需為未合作過之新增名單)。主動與地方衛生局及牙醫師公會會商建置區域內特殊需求者醫療轉診制度,增進合作醫院間交流活動。(本部特殊需求者牙科醫療服務轉介單如附件 2-4)。
\1 與身心障礙福利機構、特殊教育機構或特教班、聯合評估中心或發展遲緩療育機構、老人福利機構、長照機構或身心障礙團體等合作,提供口腔保健衛教、口腔狀況檢查及簡單診療服務,至少 6 場。
\1 至少 1 位牙醫師完成當年度特殊需求者牙科醫療服務示範中心(e.g., 雙和醫院)舉辦之牙醫師培訓計畫(需有證書)。
\1 每季(4 月、7 月及 10 月)15 日及 112 年 1 月 6 日前以電子郵件方式,填報前一季特殊需求者接受牙科醫療服務累計表(附件3)。


\end{outline}

\subsection{過去 3 年計畫執行情形及未來 3 年執行目標}
摘要表(如附件1、6)、衛生局指定辦理身心障礙者特別門診之資格證明文件(如公文,並得為影、複本)。
\subsection{品質和成效}

\subsection{預期效益}


\section{後續發展或推廣}
台灣身心障礙者口腔醫學會」專科醫師
詳述計畫執行結束後之後續規劃,以達獎助後之永續 責任,並可嘗試建立標準模式,提供其他地區標竿參考。

\subsection{x醫院牙醫醫療服務(以下稱到院牙醫服務)}
\label{dent}

\subsubsection{x特需中心}

\begin{outline}

%\1 衛生福利部「特殊需求者牙科醫療服務示範中心獎勵計畫」之醫院 
%or 「特殊需求者牙科醫療服務示範中心獎助計畫」及「特殊需求者牙科醫療服務獎助計畫」申請補助作業
\1 初級照護院所:
台北市立萬芳醫院-委託
財團法人私立台北醫學大
學辦理(醫事機構代號1301200010),審核已通過1110101至1111231。
% https://cda.org.tw/cda/public_medical_institution_type_b_search_result.jsp
%(1)申請書格式如【附件3】。 %(2)身心障礙教育訓練之學分證明影本。
%(3)牙醫師證書正反面影本一份。
\2 醫院資格
%    \3 可施行鎮靜麻醉之醫療院所及提供完備醫療之醫護人員。
    \3 設備需求:牙科門診應有急救設備、氧氣設備。
    %、麻醉機、心電 圖裝置(Monitor,包括血壓、脈搏、呼吸數之監測、血氧濃度pulse oximeter)、無障礙空間及設施(詳【附表】)。
%    \3 需1位以上具有從事相關工作經驗之醫師。
\2 醫師資格
    \3 %2位以上具有從事相關工作經驗之醫師,負責醫師自執業執照取得後滿5年以上之臨床經驗,其他醫師
    自執業執照取得後滿1年以上之臨床經驗之醫師。
    \3 經身心障礙口腔醫療教育課程(詳\ref{certificate})訓練合格。

\1 進階照護院所 %進階照護院所:口腔醫療與保健推廣計畫書
\2 醫院資格
    \3 可施行鎮靜麻醉之醫療院所及提供完備醫療之醫護人員。
    \3 設備需求:牙科門診應有急救設備、氧氣設備、麻醉機、心電 圖裝置(Monitor,包括血壓、脈搏、呼吸數之監測、血氧濃度pulse oximeter)、無障礙空間及設施(詳【附表】)。
%    \3 需2 位以上具有從事相關工作經驗之醫師。
\2 醫師資格
    \3 2位以上具有從事相關工作經驗之醫師,負責醫師自執業執照取得後滿5年以上之臨床經驗,其他醫師自執業執照取得後滿1年以上之臨床經驗。
    \3 經身心障礙口腔醫療教育課程(詳\ref{certificate})訓練合格。
\end{outline}

\subsubsection{x申請方式}