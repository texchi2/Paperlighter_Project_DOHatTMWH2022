\begin{abstract}
摘要:
「身心障礙者口腔照護指導員」訓練課程,多年的臨床工作,許多患者朋友從自行前來就醫,到現在由全家總動員推著輪椅來就醫,生活打理也漸漸需要他人代勞,甚至基本的口腔衛生照護,也面臨須由家人或照護者協助完成;平常在醫學中心,時常會有加護病房住院患者或化療患者牙科會診需求,臥床患者或這些有全身性疾病患者,經常面臨因口腔照護不佳造成牙痛或感染的困擾。

先申請特需中心,然後牙科部專任醫師才具有到宅牙醫服務的資格,當然也需要身障照護的學分。到宅小組需要一名醫師、一名護理師,攜帶必要設備,每一名案主收案都事先申請核可
%Using \LaTeX{} to write papers is concise and convenient. However, for writing in life, complicated \LaTeX{} style-files (e.g., elegantpaper) are difficult to access, or submission style-files (e.g., journal or conference) are not free indeed. To tackle these problems and satisfy an elegant and straightforward scientific writing, \textbf{paperlighter.sty}, a one-column style-file, is designed. This document is edited from icml2022.sty and provides a basic paper template. Compared to icml2022.sty, paperlighter.sty contain fewer operations, reducing adjustment while keep graceful. \textbf{\textit{Notably, the paper's main content only describes the format of icml2022.sty. We place the content to show the actual effect of paperlighter.sty.}}
\end{abstract}