\begin{abstract}
%摘要:主旨
臺北市立萬芳醫院牙科部設置特殊需求者牙科門診(以下稱特需中心),每週門診14診次,優先服務身心障礙者,牙科櫃台附有身心障礙服務窗口,設置志工,與負責身心障礙門診之牙科輔助人員共同幫忙引導、溝通、協助患者就醫。候診區及初診區均有輪椅無障礙空間。
%先申請特需中心,然後牙科部專任醫師才具有到宅牙醫服務的資格,當然也需要身障照護的學分。到宅小組需要一名醫師、一名護理師,攜帶必要設備,每一名案主收案都事先申請核可
特殊需求者的口腔照護,因日常生活打理由照顧者代勞,往往無法顧及基本的口腔衛生照護,在宅臥床患者更面臨齲齒、牙周病細菌感染的重大危機,因其就醫之阻礙甚多,四、五層樓之公寓往往沒有電梯。
特需中心配合長照2.0政策,規畫居家牙醫醫療服務。
%Using \LaTeX{} to write papers is concise and convenient. However, for writing in life, complicated \LaTeX{} style-files (e.g., elegantpaper) are difficult to access, or submission style-files (e.g., journal or conference) are not free indeed. To tackle these problems and satisfy an elegant and straightforward scientific writing, \textbf{paperlighter.sty}, a one-column style-file, is designed. This document is edited from icml2022.sty and provides a basic paper template. Compared to icml2022.sty, paperlighter.sty contain fewer operations, reducing adjustment while keep graceful. \textbf{\textit{Notably, the paper's main content only describes the format of icml2022.sty. We place the content to show the actual effect of paperlighter.sty.}}
\end{abstract}