\section{依據}
全民健康保險會(以下稱健保會)協定年度醫療給付費用總額事項辦理「111年全民健康保險牙醫門診總額特殊醫療服務計畫」(以下稱本計畫)。

\section{目的} 旨在提升牙醫醫療服務品質,加強提供特定身心障礙者牙醫醫療服務。

%\section{計畫種類}
\section{特殊需求者牙科醫療服務計畫}

\subsection{醫院牙醫醫療服務(以下稱到院牙醫服務)}
\label{dent}

\subsubsection{特需中心}

\begin{outline}

%\1 衛生福利部「特殊需求者牙科醫療服務示範中心獎勵計畫」之醫院 
%or 「特殊需求者牙科醫療服務示範中心獎助計畫」及「特殊需求者牙科醫療服務獎助計畫」申請補助作業
\1 初級照護院所
\1 進階照護院所
\2 醫院資格
    \3 可施行鎮靜麻醉之醫療院所及提供完備醫療之醫護人員。
    \3 設備需求:牙科門診應有急救設備、氧氣設備、麻醉機、心電 圖裝置(Monitor,包括血壓、脈搏、呼吸數之監測、血氧濃度pulse oximeter)、無障礙空間及設施(詳【附表】)。
%    \3 需2 位以上具有從事相關工作經驗之醫師。
\2 醫師資格
    \3 2位以上具有從事相關工作經驗之醫師,負責醫師自執業執照取得後滿5年以上之臨床經驗,其他醫師自執業執照取得後滿1年以上之臨床經驗。
    \3 身心障礙口腔醫療教育課程(詳\ref{certificate}醫師資格)。
\end{outline}

\subsubsection{申請方式}
\begin{outline}

\1 全民健康保險牙醫門診總額特殊醫療服務計畫牙醫醫療服務加入申請書(院所內服務)【附件3】
\1 全民健康保險牙醫門診總額特殊醫療服務計畫院所申請流程圖【附件7】
\1 牙科醫療院所友善醫療環境評量表【附表】

\end{outline}

%\subsection{醫療團牙醫醫療服務}
%x (七)(到身心障礙福利機構)
%因本計畫醫療團成立的規定,護理之家僅限定照護司擇定之2家,目前無法新增。另醫療團主要服務機構內符合本計畫特定障別之患者,因其困難外出就醫,故組成醫療團至機構內服務。而貴院所屬護理之家已在萬芳醫院內,就醫方便,建議循一般就診方式處理。

%以醫療團為單位,申請時應檢附下列資料:(含特定需求者牙醫醫療服務){申請書格式【附件5】、牙醫師公會評估表【附件6】、護理之家(由衛生福利部護理及健康照護司擇定)之立案證明、同意函、簡介、收容對象名冊、口腔狀況、牙科設備、 醫師服務排班表、牙科治療計畫、維護計畫、口腔衛生計畫、經費評估、牙醫師證書正反面影本乙份等}

\subsection{居家牙醫醫療服務(以下稱到宅牙醫服務)}
% (八)居家牙醫醫療服務 (p15-18)

\subsubsection{服務對象:}
\begin{outline}
\0 限居住於住家(不含照護機構)的牙醫特殊需求病人,需符合下列條件之一
    \1 全民健康保險居家醫療照護整合計畫之居家醫療、重度居家醫療及安寧療護階段之病人,且有明確之牙醫醫療需求。
    \1 出院準備及追蹤管理費(02025B)申報病人,且有明確之牙醫醫療需求。
    \1 特定身心障礙者,清醒時百分之五十以上活動限制在床上或椅子上,且有明確之牙醫醫療需求。前述特定身心障礙者之障礙類別包含:
        \2 肢體障礙(限腦性麻痺、腦傷及脊髓損傷之中度肢體障礙、及重度以上肢體障礙)
        \2 重度以上視覺障礙、重度以上重要器官失去功能
        \2 中度以上之植物人、智能障礙、自閉症、精神障礙、失智症、頑固性(難治型)癲癇
        \2 因罕見疾病而致身心功能障礙、染色體異常、發展遲緩兒童
        \2 其他經主管機關認定之障礙(須為新制評鑑為第1、4、5、6、7類者)
        \2 多重障礙(或同時具備二種及二種以上障礙類別)

    \1 「失能老人接受長期照顧服務補助辦法」之補助對象(以下稱失能老人),並為各縣市長期照顧管理中心之個案,且因疾病、傷病長期臥床的狀態,清醒時百分之五十以上活動限制在床上或椅子上,行動困難無法自行至醫療院所就醫之病人。

\end{outline}

\subsubsection{服務流程【附件18】}
\label{certificate}


% Set spacing between columns
\setlength{\tabcolsep}{8pt}

% Set the width of each column
\begin{longtable}{p{1.3in}p{4.8in}}

%\color{OliveGreen}{收案條件}
% & 限居住於住家(不含照護機構)且符合下列條件之一者:全民健康保險居家醫療照護整合計畫之居家醫療病人(重度或極重度特定身心障礙者)、安寧療護階段之病人、出院準備及追蹤管理費(02025B)申報病人、失能老人(為各縣市長期照顧管理中心之個案)。 \\
\color{OliveGreen}{事前申請}
 & 首次訪視申請表【附件15】、後續追蹤治療計畫【附件16】均須註明後送轉診醫院。經牙醫全聯會核可後,始得至案家提供牙醫醫療服務。 \\
 
\color{OliveGreen}{醫師資格}
& 已參加本計畫照護院所(詳\ref{dent})之專任醫師。
自執業執照取得後滿1年以上臨床經驗之醫師。每位醫師首次加入本計畫,須接受6學分以上身心障礙口腔醫療業務等相關之基礎教育訓練。加入計畫後,每年須再接受4學分以上之身心障礙口腔醫療業務相關之再進修教育課(每年再進修課程不得重複);本計畫之醫師須累積七年以上且超過 30(含)學分後,得繼續執行計畫,惟課程皆須由中華牙醫學會或牙醫全聯會認證通過。 \\

\color{OliveGreen}{到宅小組}
 & 一名醫師搭配一名護理師出診; 須有熟悉該患者狀況的人員(家屬)陪同就診。護理師至病人住家提供醫療服務,則須依法令規定事前報經當地衛生主管機關核准。因符合醫師法所稱應邀出診,不需經事先報准。\\
 
\color{OliveGreen}{就診文書}
& 醫師應於院所製作電子病歷留存,且須將病人身分影印本及計畫所須之證明文件,黏貼於病歷首頁後掃瞄為電子檔 留存,以備查驗。就診紀錄【附件12】應詳實紀錄。侵入性治療應取得病人家屬或監護人之書面同意書。\\
 
\color{OliveGreen}{攜帶設備}
& 生理監測器(血壓、血氧)、攜帶式洗牙機、攜帶式強力抽吸設備與抽痰管、牙科治療器械、有效的急救設備(人工氣道 laryngeal mask airway, LMA)、氧氣設備(含氧氣幫浦、氧氣筒須有節流裝置、氧氣面罩等)、急救藥品、開口器及健保卡讀寫卡設備等相關物品。\\

\color{OliveGreen}{照護內容}
& 基於安全考量,居家牙醫醫療服務時,以提供牙周病緊急處理、牙周敷料、牙結石清除、牙周暨齲齒控制基本處置、塗氟、(非)特定局部治療、簡單性拔牙及單面蛀牙填補等服務為限。居家牙醫醫療服務給付項目及支付標準詳【附件17】。\\

\color{OliveGreen}{病人安全}
& 診療期間隨時注意病人之生理及心理狀況。若遇臨時緊急狀況或危急情形,應初步保護呼吸道,並立刻和計畫中最近的後送醫院聯絡,進行緊急醫療及後送程序。\\


\color{OliveGreen}{感染管制}
& 牙醫醫療服務應符合「牙醫巡迴醫療、特殊醫療、矯正機關之牙醫服務感染管制 SOP 作業細則」【附件14】。設備之維護、清潔保養及醫療廢棄物由醫療院所依相關法規妥善處理。\\

\color{OliveGreen}{部分負擔}
& 依牙醫門診基本部分負擔計收。\\

\color{OliveGreen}{醫療費用}
& 特約醫事服務機構執行本計畫之醫療費用應按月申報給付項目【附件17】,並於門診醫療服務點數清單依下列規定填報「案件分類」及「特定治療項 目代號」欄位,案件分類 16,特定治療項目代號(一)請依病人類 別填報【極重度FS、重度FY、中度 L4、發展遲緩兒童LE、失能老人L2、居整病人LC、出院準備LD、腦傷及脊髓損傷之中度肢體障礙:LJ】。醫令類別為
「4 不得另計價之藥品、檢驗(查)、診療項目或材料」。\\



\color{OliveGreen}{相關規範}
%\begin{outline}
& 個案首次接受居家牙醫醫療服務(含訪視)前,牙醫師須檢送申請
表【附件15】至牙醫全聯會,由牙醫全聯會於每月 20 日前將核可 名單函送保險人分區業務組備查。個案於首次接受居家牙醫醫療服 務(含訪視)後,須於次月 20 日前檢送病人之「口腔醫療需求評 估及治療計畫」【附件16】,正本送所屬保險人分區業務組、副本送 牙醫全聯會備查。經催繳三個月內仍未改善者,經保險人分區業務 組及牙醫全聯會確認,得暫停執行居家牙醫醫療服務。


%\end{outline}

\end{longtable}

\section{牙醫特殊計畫承辦窗口}
\noindent 社團法人中華民國牙醫師公會全國聯合會\\
牙醫特殊計畫承辦人朱智華\\
104台北市中山區復興北路420號10樓\\
電話:02-25000133 ext. 262\\
傳真:02-25000126\\
uase@cda.org.tw\\