% SWOT: Streghts, Weaknesses, Opportunities, Threats
% https://gist.github.com/ricardogarfe/4563453


\section{SWOT Matrix - \emph{(Strengths, Weaknesses, Opportunities, Threats)}}

% pentagon -> hexagon
\begin{tikzpicture}[
    hexagon/.style={%
        shape=regular polygon, regular polygon sides=6, minimum size=7.3cm, inner
        sep=-1mm, draw, fill=gray!10 %DarkSeaGreen!75!yellow
    }, font=\scriptsize\sffamily, thick
]

% \draw[help lines] (-16,-16) grid (16,16);
\filldraw[thin,gray,fill=gray!25] (-8,-8) rectangle (8,8);
\filldraw[thin,gray,fill=white] (-7.15,-7.15) rectangle (7.15,7.15);
\draw[thin,gray] (7.15,7.15)--(8,8) (-7.15,7.15)--(-8,8) (-7.15,-7.15)--(-8,-8)
(7.15,-7.15)--(8,-8);

% Strengths
% pentagon
\draw[thin, green] (-0.025,0.025)--(-7.05,0.025)--(-0.025,7.05)--cycle;

%\node[pentagon ,rotate=45] 
%\polygon[rotate=45]{6} 
\node[hexagon, rotate=45] at (-3.75,3.75) {
    \begin{varwidth}{\linewidth}
        \begin{itemize}[leftmargin=*,noitemsep]
            \item 社會責任與醫院評鑑
            %Technical and business expertise
            \item 守護文山新店深坑地區 %Domestic market orientation
            \item 偏鄉服務及特需中心 %初級
            %Stable management team
            \item 牙科部財務健全
            %Financial stability
            %\item Acquisition capabilities            
            \item 資深特需師資(四位) %七年30學分終身會員 %Economies of scale
            \item 複製北醫附醫五年到宅經驗 %Training programs
            %\item Loyalty and retention
        \end{itemize}
    \end{varwidth}
};
\draw (-2,2) node[rotate=45] {\large\textbf{Strengths}};

% Weaknesses
\draw[thin, brown!20!red] (0.025,0.025)--(7.05,0.025)--(0.025,7.05)--cycle;
\node[hexagon,rotate=-45] at (3.75,3.75) {
    \begin{varwidth}{\linewidth}
        \begin{itemize}[leftmargin=*,noitemsep]
            \item 初期經費與設備投資 %Centralized decisions
            \item  醫師、專責護士與人力規畫 %Accounts cross-selling
            \item 到宅牙醫行政文書
             %Marketing capabilities
            %\item 到宅醫療收入與時間成本 %Win on price image
            %\item 到宅交通費用 %BPO market
            %\item 特需案例來源
            %\item No differentiation
        \end{itemize}
    \end{varwidth}
};
\draw (2,2) node[rotate=-45] {\large\textbf{Weaknesses}};

% Opportunities
\draw[thin, green!50!blue] (-0.025,-0.025)--(-7.05,-0.025)--(-0.025,-7.05)--cycle;
%135
\node[hexagon,rotate=-45] at (-3.75,-3.75) {
    \begin{varwidth}{\linewidth}
        \begin{itemize}[leftmargin=*,noitemsep]
            \item  原PGY醫師(1人/月)支援連江縣立醫院 %Marketing push
            \item =>社區牙科訓練改為到宅牙醫服務 %Adding BPO capabilities
            \item 招募護士加入排班 %到宅醫師(二名排班) \item 到宅護士(二名排班)
            \item 牙科部祕書兼到宅牙醫行政
            \item 收取交通費/院方公務車支援 %Pricing structure
            \item 案例來源: 社區醫療部、出院準備服務 %Business process approach
            %\item %Annuity engagement
        \end{itemize}
    \end{varwidth}
};
\draw (-2,-2) node[rotate=-45] {\large\textbf{Opportunities}};

% Threats
\draw[thin, brown] (0.025,-0.025)--(7.05,-0.025)--(0.025,-7.05)--cycle;
\node[hexagon,rotate=45] at (3.75,-3.75) {
    \begin{varwidth}{\linewidth}
        \begin{itemize}[leftmargin=*,noitemsep]
            \item 到宅醫療收入不敷時間成本 %Profitability losses
            \item 收取到宅交通費形成門檻 %BPO market
            \item 特需案例來源
            
            %\item High-risk deals
            %\item Image change inability
            %\item Degree of automation
        \end{itemize}
    \end{varwidth}
};
\draw (2,-2) node[rotate=45] {\large\textbf{Threats}};
%
\draw(0,-7.55) node {\Large EXTERNAL};
\draw(0,7.55) node {\Large INTERNAL};
\draw(-7.55,0) node[rotate=90] {\Large POSITIVE};
\draw(7.55,0) node[rotate=270] {\Large NEGATIVE};
\draw[green](-0.6,0.6) node {\Huge\textbf{S}}; 
\draw[brown!20!red](0.6,0.6) node {\Huge\textbf{W}};
\draw[green!50!blue](-0.6,-0.6) node {\Huge\textbf{O}};
\draw[brown](0.6,-0.6) node {\Huge\textbf{T}};
\end{tikzpicture}
  
  % treatment planning
% U:如何善用每個優勢? (How can we Use each Strength?)
%S:如何停止每個劣勢? (How can we Stop each Weakness?)
%E:如何成就每個機會? (How can we Exploit each Opportunity?)
%D:如何抵禦每個威脅? (How can we Defend against each Threat?)
