% SWOT: Streghts, Weaknesses, Opportunities, Threats
% https://gist.github.com/ricardogarfe/4563453


\section{SWOT Matrix - \emph{(Strengths, Weaknesses, Opportunities, Threats)}}

% pentagon -> hexagon
\begin{tikzpicture}[
    hexagon/.style={%
        shape=regular polygon, regular polygon sides=6, minimum size=7.3cm, inner
        sep=-1mm, draw, fill=gray!10 %DarkSeaGreen!75!yellow
    }, font=\scriptsize\sffamily, thick
]

% \draw[help lines] (-16,-16) grid (16,16);
\filldraw[thin,gray,fill=gray!25] (-8,-8) rectangle (8,8);
\filldraw[thin,gray,fill=white] (-7.15,-7.15) rectangle (7.15,7.15);
\draw[thin,gray] (7.15,7.15)--(8,8) (-7.15,7.15)--(-8,8) (-7.15,-7.15)--(-8,-8)
(7.15,-7.15)--(8,-8);

% Strengths
% pentagon
\draw[thin, green] (-0.025,0.025)--(-7.05,0.025)--(-0.025,7.05)--cycle;

%\node[pentagon ,rotate=45] 
%\polygon[rotate=45]{6} 
\node[hexagon, rotate=45] at (-3.75,3.75) {
    \begin{varwidth}{\linewidth}
        \begin{itemize}[leftmargin=*,noitemsep]
            \item 社會責任與醫院評鑑
            %Technical and business expertise
            \item 守護文山新店深坑地區 %Domestic market orientation
            \item 偏鄉服務及特需中心 %初級
            %Stable management team
            \item 牙科部財務健全
            %Financial stability
            %\item Acquisition capabilities            
            \item 資深特需師資(四位) %七年30學分終身會員 %Economies of scale
            \item 複製北醫附醫五年到宅經驗 %Training programs
            %\item Loyalty and retention
        \end{itemize}
    \end{varwidth}
};
\draw (-2,2) node[rotate=45] {\large\textbf{Strengths}};

% Weaknesses
\draw[thin, brown!20!red] (0.025,0.025)--(7.05,0.025)--(0.025,7.05)--cycle;
\node[hexagon,rotate=-45] at (3.75,3.75) {
    \begin{varwidth}{\linewidth}
        \begin{itemize}[leftmargin=*,noitemsep]
            \item 初期經費與設備投資 %Centralized decisions
            \item  醫師、專責護士與人力規畫 %Accounts cross-selling
            \item 到宅牙醫行政文書
             %Marketing capabilities
            %\item 到宅醫療收入與時間成本 %Win on price image
            %\item 到宅交通費用 %BPO market
            %\item 特需案例來源
            %\item No differentiation
        \end{itemize}
    \end{varwidth}
};
\draw (2,2) node[rotate=-45] {\large\textbf{Weaknesses}};

% Opportunities
\draw[thin, green!50!blue] (-0.025,-0.025)--(-7.05,-0.025)--(-0.025,-7.05)--cycle;
%135
\node[hexagon,rotate=-45] at (-3.75,-3.75) {
    \begin{varwidth}{\linewidth}
        \begin{itemize}[leftmargin=*,noitemsep]
            \item  原PGY醫師(1人/月)支援連江縣立醫院 %Marketing push
            \item =>社區牙科訓練改為到宅牙醫服務 %Adding BPO capabilities
            \item 招募護士加入排班 %到宅醫師(二名排班) \item 到宅護士(二名排班)
            \item 牙科部祕書兼到宅牙醫行政
            \item 收取交通費/院方公務車支援 %Pricing structure
            \item 案例來源: 社區醫療部、出院準備服務 %Business process approach
            %\item %Annuity engagement
        \end{itemize}
    \end{varwidth}
};
\draw (-2,-2) node[rotate=-45] {\large\textbf{Opportunities}};

% Threats
\draw[thin, brown] (0.025,-0.025)--(7.05,-0.025)--(0.025,-7.05)--cycle;
\node[hexagon,rotate=45] at (3.75,-3.75) {
    \begin{varwidth}{\linewidth}
        \begin{itemize}[leftmargin=*,noitemsep]
            \item 到宅醫療收入不敷時間成本 %Profitability losses
            \item 收取到宅交通費形成門檻 %BPO market
            \item 特需案例來源
            
            %\item High-risk deals
            %\item Image change inability
            %\item Degree of automation
        \end{itemize}
    \end{varwidth}
};
\draw (2,-2) node[rotate=45] {\large\textbf{Threats}};
%
\draw(0,-7.55) node {\Large EXTERNAL};
\draw(0,7.55) node {\Large INTERNAL};
\draw(-7.55,0) node[rotate=90] {\Large POSITIVE};
\draw(7.55,0) node[rotate=270] {\Large NEGATIVE};
\draw[green](-0.6,0.6) node {\Huge\textbf{S}}; 
\draw[brown!20!red](0.6,0.6) node {\Huge\textbf{W}};
\draw[green!50!blue](-0.6,-0.6) node {\Huge\textbf{O}};
\draw[brown](0.6,-0.6) node {\Huge\textbf{T}};
\end{tikzpicture}
  
  % treatment planning
% U:如何善用每個優勢? (How can we Use each Strength?)
%S:如何停止每個劣勢? (How can we Stop each Weakness?)
%E:如何成就每個機會? (How can we Exploit each Opportunity?)
%D:如何抵禦每個威脅? (How can we Defend against each Threat?)

\begin{outline}

% 零基預算 Zero-based budgeting system
% 成本中心
\0 經費概算:
\1 人事(聘用人力學經歷) %Required staffing to complete department functions; What staffing would be required at what experience levels & salaries
\1 設備 %Required equipment/software/hardware necessary to complete the function
\1 出診交通費 %Required overhead costs needed to complete the function
\1 耗材 %Required overhead costs needed to complete the function
\1 教育訓練
\1 衛教宣導

\0 預期效益及效益指標(key performance index, KPI) %Current marketing efforts and their efficiency
\1 每週出診數、人次
\2 健保收入,交通費收入。轉診加成?
%(一)	預計提供高負荷家庭照顧者個案服務○○○人。
\2 每年預計提供宅照顧技巧○○人/○○人次
\1 每年預計辦理長照家庭照顧者之長照知識或照顧技能訓練○○○場,服務○○人及○○人次

%(四)	預計拓展○○照顧資訊小站、○○照顧支持小站。
%(五)	預計辦理○○場教育訓練,受益人數為○○人次。
\1 預計辦理○○場宣導活動,擴及人數為○○人次

\end{outline}

2022年籌備期程(Gantt chart)\\
\begin{tabular}{|c|ccccccccc|}
\hline
 工作項目 & 四月 & 五月  & 六月 & 七月  & 七月  & 七月  & 七月  & 七月  & 七月  \\
\hline
 聘用護理人員    &  & &  & $\longrightarrow$ & $\longrightarrow$ & $\longrightarrow$ & $\longrightarrow$ & $\longrightarrow$ & $\longrightarrow$  \\
 設備採購    &  &  & $\longrightarrow$ & $\longrightarrow$ & &&&&  \\
 衛教宣導    & $\longrightarrow$ & $\longrightarrow$ & $\longrightarrow$ & $\longrightarrow$  & $\longrightarrow$ & $\longrightarrow$ & $\longrightarrow$ & $\longrightarrow$ & $\longrightarrow$ \\
 online    &  & &  & $\longrightarrow$ & $\longrightarrow$ & $\longrightarrow$ & $\longrightarrow$ & $\longrightarrow$ & $\longrightarrow$  \\
\hline
\end{tabular}

\vspace{5mm}
區域內整體服務辦理情形(盤點申請區域服務資源及說明單位組織量能、目前長照服務推動情形外,請加強敘明目前及未來辦理家庭照顧者服務情形):
社區資源開發整合
%https://www.nhi.gov.tw/BBS_Detail.aspx?n=73CEDFC921268679&sms=D6D5367550F18590&s=DC5108C6E3DD21D0

特需中心公告名單(2022年)
\begin{outline}
%\0 

\1 到宅牙醫服務
\2 全國136名醫師
\2 臺北市52名醫師
\2 文山區計有(德威牙醫診所) 陳義聰、蕭雅純二名醫師

\1 特需服務院所(到院牙醫服務)
\2 文山區計有14家院所
\end{outline}

臺北市文山區特須牙醫院所名單\\
\begin{tabularx}{1.0625\textwidth}{|c|p{3.2cm}|c|l|}
\hline
臺北&	童芯牙醫診所&	02-29366707 &	臺北市文山區久康街24巷7號1樓\\
\hline
臺北&	全家牙醫診所&	02-22346393 &	臺北市文山區木柵路2段143號\\
\hline
臺北 &	欣美牙醫診所 &	02-22343708 &	臺北市文山區木柵路3段10號(1、2樓)\\
\hline
臺北 &	永麗牙醫診所 &	02-29372227 &	臺北市文山區木新路3段111號(1、2樓)\\
\hline
臺北 &	天丞牙醫診所 &	02-29378380 &	臺北市文山區木新路3段325號\\
\hline
臺北 &	驊陽牙醫診所 &	02-86610188 &	臺北市文山區忠順街1段26巷32號\\
\hline
臺北 &	景華牙醫診所 &	02-29304167 &	臺北市文山區景華街122號1樓\\
\hline
臺北 &	德在牙醫診所 &	02-29338248 &	臺北市文山區景福街30號1樓\\
\hline
臺北 &	雅田牙醫診所 &	02-29352838 &	臺北市文山區景興路119號1樓\\
\hline
臺北 &	家樂牙醫診所 &	02-82301197 &	臺北市文山區萬安街33號\\
\hline
臺北 &	德威牙醫診所 &	02-29328281 &	臺北市文山區興隆路1段72、74號、70巷1-1號2樓\\
\hline
臺北 &	臺北市立萬芳醫院-委託財團法人臺北醫學大學辦理
&	02-29307930 &	臺北市文山區興隆路3段111號\\
\hline
臺北 &	廣泉牙醫診所 &	02-29343572 &	臺北市文山區興隆路3段41號\\
\hline
臺北 &	德全牙醫診所 &	02-29341569 &	臺北市文山區羅斯福路6段248號1樓\\
\hline
\end{tabularx}

% https://ceriniandassociates.com/news-feed/2020/07/13/zero-based-budgeting/
%One tool which can be helpful to practices of all sizes can be to periodically implement a “zero-based” budgeting approach. Zero-based budgeting was first developed and implemented in the 1970s and then driven to extremes by investment firms such as 3G Capital in the 2000s as a tool to cut costs wherever possible and to extremes, such as focusing on such minute details as the number of pages printed and photocopies made by employees. As a result, zero-based budgeting has a reputation as an “austerity” measure and a tool used to cut costs to the bone. This zealous approach to the ideals may have its place in many organizations, however, the original ideals and principles of the system should be part of any organization’s planning toolbox from time-to-time.

%In a “zero-based” budgeting system, departments should look periodically at each year and budget as if starting from a zero-dollar budget allocation, rather than just what was spent in the past. The real point of the exercise is to take a top-down approach and determine if all spend in any given department is required to fulfill the functions of this department. The additional scrutiny can be used to uncover potential inefficiencies, over or understaffing in departments, discover potential synergies between departments, and empower department heads to perform an overall review of their operations.

%In implanting a zero-based budgeting system, a healthcare organization should take the following steps:
%1.) Start a baseline zero for all departments; prior-year spending does not matter.
%2.) Evaluate every cost area within the department (in conjunction with the department head). This evaluation should include:
%a.) Required staffing to complete department functions
%b.) Required equipment/software/hardware necessary to complete the function
%c.) Required overhead costs needed to complete the function

%When evaluating, consider how one would start *** a brand-new department from scratch. 
%What staffing would be required at what experience levels & salaries; what costs are required to perform the necessary operations? 
%The critical eye here is necessary and often best done in a collaborative effort with both an insider (someone in the department) and an outsider (someone outside the department). 外部委員

%3.) Justify the spending in these above areas and try to identify cost savings. Some examples may include:
%a.) Looking at the cost of outsourcing billing vs. proving billing services in-house
%b.) Staffing levels of reception/administrative staff
%c.) Current marketing efforts and their efficiency