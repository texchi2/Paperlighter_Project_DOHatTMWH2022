\documentclass[11pt,twoside]{article}
\maxdeadcycles=1000 % Output loop---200 consecutive dead cycles.

\usepackage{paperlighter}

% Remove page numbering
\pagenumbering{gobble}

\usepackage{longtable, setspace, outlines}
 % for OliveGreen
\definecolor{OliveGreen}{rgb}{0,0.6,0}
% Set link colors
%\usepackage[rgb,dvipsnames]{xcolor}
%\hypersetup{colorlinks=true, linkcolor=RoyalBlue, urlcolor=RoyalBlue}

\usepackage{pdfpages} % for includepdf

\usepackage{selinput}
%\usepackage[margin=2cm]{geometry}
\usepackage{enumitem,varwidth}
%\usepackage[svgnames]{xcolor}
\usepackage{tikz} % for SWOT
\usetikzlibrary{shapes.geometric}

% Recommended, but optional, packages for figures and better typesetting:
\usepackage{microtype}
\usepackage{graphicx}
\usepackage{subfigure}
\usepackage{booktabs} % for professional tables

% Attempt to make hyperref and algorithmic work together better:
\newcommand{\theHalgorithm}{\arabic{algorithm}}


% For theorems and such
\usepackage{amsmath}
\usepackage{amssymb}
\usepackage{mathtools}
\usepackage{amsthm}

% if you use cleveref..
\usepackage[capitalize,noabbrev]{cleveref}

%%%%%%%%%%%%%%%%%%%%%%%%%%%%%%%%
% THEOREMS
%%%%%%%%%%%%%%%%%%%%%%%%%%%%%%%%
\theoremstyle{plain}
\newtheorem{theorem}{Theorem}[section]
\newtheorem{proposition}[theorem]{Proposition}
\newtheorem{lemma}[theorem]{Lemma}
\newtheorem{corollary}[theorem]{Corollary}
\theoremstyle{definition}
\newtheorem{definition}[theorem]{Definition}
\newtheorem{assumption}[theorem]{Assumption}
\theoremstyle{remark}
\newtheorem{remark}[theorem]{Remark}

% Todonotes is useful during development; simply uncomment the next line
%    and comment out the line below the next line to turn off comments
%\usepackage[disable,textsize=tiny]{todonotes}
\usepackage[textsize=tiny]{todonotes}

% Chinese
\usepackage{xeCJK} % for Chinese, compiling by XeLaTex
\usepackage{indentfirst}
\setlength{\parindent}{2em}  % setting the indentation to be two Chinese characters size.
\usepackage{fontspec} %設定字體
% Fandol font (the default)  not shown "內"
\setCJKmainfont{[Iansui094-Regular.ttf]}
\setCJKsansfont{[Iansui094-Regular.ttf]}
\setCJKmonofont{[Iansui094-Regular.ttf]}
%\setCJKmainfont{AR PL UMing TW MBE} % AR PL UMing TW MBE or "UKai" https://www.overleaf.com/learn/latex/Questions/Which_OTF_or_TTF_fonts_are_supported_via_fontspec%3F#Chinese
%BiauKai} %標楷體 from macOS %設定中文為系統上的字型,而英文不去更動,使用原TeX字型
%\setCJKmainfont[Vertical=RotatedGlyphs]{AR PL UMing TW MBE}


\slimtitle{特需中心} %Paperlighter Example}
\slimauthor{萬芳醫院牙科部}


\begin{document}

\lightertitle{\hspace{3.5cm} 臺北市立萬芳醫院牙科部

\hspace{2.5cm}特殊需求者口腔醫療與保健推廣計畫}
% 特殊需求者牙醫醫療服務
%居家牙醫醫療服務計畫
% 進階照護院所:口腔醫療與保健推廣計畫書

\lighterauthor{祁力行$^{\dagger}$, 李勝揚$^{\ddagger}$}

\lighteraddress{$^\dagger$}{臺北市立萬芳醫院牙科部口腔顎面外科}
\lighteraddress{$^\ddagger$}{臺北市立萬芳醫院牙科部}


\lighteremail{110050@w.tmu.edu.tw}


%\begin{abstract}
%摘要:主旨
臺北市立萬芳醫院牙科部設置特殊需求者牙科門診(以下稱特需中心),每週門診14診次,優先服務身心障礙者,牙科櫃台附有身心障礙服務窗口,設置志工,與負責身心障礙門診之牙科輔助人員共同幫忙引導、溝通、協助患者就醫。候診區及初診區均有輪椅無障礙空間。
%先申請特需中心,然後牙科部專任醫師才具有到宅牙醫服務的資格,當然也需要身障照護的學分。到宅小組需要一名醫師、一名護理師,攜帶必要設備,每一名案主收案都事先申請核可
特殊需求者的口腔照護,因日常生活打理由照顧者代勞,往往無法顧及基本的口腔衛生照護,在宅臥床患者更面臨齲齒、牙周病細菌感染的重大危機,因其就醫之阻礙甚多,四、五層樓之公寓往往沒有電梯。
特需中心配合長照2.0政策,規畫居家牙醫醫療服務。
%Using \LaTeX{} to write papers is concise and convenient. However, for writing in life, complicated \LaTeX{} style-files (e.g., elegantpaper) are difficult to access, or submission style-files (e.g., journal or conference) are not free indeed. To tackle these problems and satisfy an elegant and straightforward scientific writing, \textbf{paperlighter.sty}, a one-column style-file, is designed. This document is edited from icml2022.sty and provides a basic paper template. Compared to icml2022.sty, paperlighter.sty contain fewer operations, reducing adjustment while keep graceful. \textbf{\textit{Notably, the paper's main content only describes the format of icml2022.sty. We place the content to show the actual effect of paperlighter.sty.}}
\end{abstract} % no more
%\input{content/format}
\section{依據}
全民健康保險會(以下稱健保會)協定年畫畫度醫療給付費用總額事項辦理「全民健康保險牙醫門診總額特殊醫療服務計畫」(以下稱本計畫)。

\section{目的} 旨在提升牙醫醫療服務品質,加強提供特定身心障礙者牙醫醫療服務。

\section{計畫種類}

\subsection{牙醫醫療服務照護醫院(到院)}
%(特需中心: 
\begin{outline}

\1 衛生福利部「特殊需求者牙科醫療服務示範中心獎勵計畫」之醫院 
%\1 初級照護院所
\1 進階照護院所
\2 醫院資格
    \3 可施行鎮靜麻醉之醫療院所及提供完備醫療之醫護人員。
    \3 設備需求:牙科門診應有急救設備、氧氣設備、麻醉機、心電 圖裝置(Monitor,包括血壓、脈搏、呼吸數之監測、血氧濃度oximeter)、無障礙空間及設施詳【附表】。
%    \3 需2 位以上具有從事相關工作經驗之醫師。
\2 醫師資格
    \3 2位以上具有從事相關工作經驗之醫師,負責醫師自執業執照取得後滿5年以上之臨床經驗,其他醫師自執業執照取得後滿1年以上之臨床經驗。
    \3 身心障礙口腔醫療教育課程(詳\ref{certificate}醫師資格)。

\0 申請方式
\1 全民健康保險牙醫門診總額特殊醫療服務計畫牙醫醫療服務加入申請書(院所內服務)【附件3】
\1 全民健康保險牙醫門診總額特殊醫療服務計畫院所申請流程圖【附件7】
\1 牙科醫療院所友善醫療環境評量表【附表】

\end{outline}

%\subsection{醫療團牙醫醫療服務}
%x (七)(到身心障礙福利機構)
%因本計畫醫療團成立的規定,護理之家僅限定照護司擇定之2家,目前無法新增。另醫療團主要服務機構內符合本計畫特定障別之患者,因其困難外出就醫,故組成醫療團至機構內服務。而貴院所屬護理之家已在萬芳醫院內,就醫方便,建議循一般就診方式處理。

%以醫療團為單位,申請時應檢附下列資料:(含特定需求者牙醫醫療服務){申請書格式【附件5】、牙醫師公會評估表【附件6】、護理之家(由衛生福利部護理及健康照護司擇定)之立案證明、同意函、簡介、收容對象名冊、口腔狀況、牙科設備、 醫師服務排班表、牙科治療計畫、維護計畫、口腔衛生計畫、經費評估、牙醫師證書正反面影本乙份等}

\subsection{居家牙醫醫療服務(到宅)}
% (八)居家牙醫醫療服務 (p15-18)

\subsubsection{服務對象}
\begin{outline}
\0 限居住於住家(不含照護機構)的特殊 需求病人,符合下列條件之一
    \1 全民健康保險居家醫療照護整合計畫之居家醫療、重度居家醫療及
    安寧療護階段之病人,且有明確之牙醫醫療需求。
    \1 出院準備及追蹤管理費(02025B)申報病人,且有明確之牙醫醫療需
求。
    \1 特定身心障礙者,清醒時百分之五十以上活動限制在床上或椅子上,
且有明確之醫療需求。前述特定身心障礙者之障礙類別包含:肢體 障礙(限腦性麻痺、腦傷及脊髓損傷之中度肢體障礙、及重度以上 肢體障礙)、重度以上視覺障礙、重度以上重要器官失去功能,以 及中度以上之植物人、智能障礙、自閉症、精神障礙、失智症、多 重障礙(或同時具備二種及二種以上障礙類別)、頑固性(難治型) 癲癇、因罕見疾病而致身心功能障礙、染色體異常、其他經主管機 關認定之障礙(須為新制評鑑為第 1、4、5、6、7 類者)或發展遲 緩兒童等。
    \1 「失能老人接受長期照顧服務補助辦法」之補助對象(以下稱失能老 人),並為各縣市長期照顧管理中心之個案,且因疾病、傷病長期臥 床的狀態,清醒時百分之五十以上活動限制在床上或椅子上,行動 困難無法自行至醫療院所就醫之病人。

\end{outline}

\subsubsection{服務流程【附件18】}
\label{certificate}
% Set spacing between columns
\setlength{\tabcolsep}{8pt}

% Set the width of each column
\begin{longtable}{p{1.3in}p{4.8in}}

%\color{OliveGreen}{收案條件}
% & 限居住於住家(不含照護機構)且符合下列條件之一者:全民健康保險居家醫療照護整合計畫之居家醫療病人(重度或極重度特定身心障礙者)、安寧療護階段之病人、出院準備及追蹤管理費(02025B)申報病人、失能老人(為各縣市長期照顧管理中心之個案)。 \\
\color{OliveGreen}{事前申請}
 & 首次訪視申請表【附件15】、後續追蹤治療計畫【附件16】均須註明後送轉診醫院。經牙醫全聯會核可後,始得 至案家提供牙醫醫療服務。 \\
 
\color{OliveGreen}{醫師資格}
& 已參加本計畫照護院所之專任醫師。
自執業執照取得後滿1年以上臨床經驗之醫師。每位醫師首次加入本計畫,須接受6學分以上身心障礙口腔醫療業務等相關之基礎教育訓練。加入計畫後,每年須再接受4學分以上之身心障礙口腔醫療業務相關之再進修教育課(每年再進修課程不得重複);本計畫之醫師須累積七年以上且超過 30(含)學分後,得繼續執行計畫,惟課程皆須由中華牙醫學會或牙醫全聯會認證通過。 \\

\color{OliveGreen}{居家小組}
 & 一名醫師搭配一名護理師出診; 須悉該患者狀況的人員(家屬)陪同就診。符合醫師法所稱應邀出診,不需經事先報准。護理師至病人住家提供醫療服務,則須依法令規定事前報經當地衛生主管機關核准。\\
 
\color{OliveGreen}{就診文書}
& 就診紀錄【附件12】應詳實紀錄,院所內須製作電子病歷留存。侵入性治療應取得病人家屬或監護人之書面同意書。\\
 
\color{OliveGreen}{攜帶設備}
& 生理監測器(血壓、血氧)、攜帶式洗牙機、攜帶式強力抽吸設備與抽痰管、牙科治療器械、有效的急救設備(人工氣道 laryngeal mask airway, LMA)、氧氣設備(含氧氣幫浦、氧氣筒須有節流裝置、氧氣面罩等)、急救藥品、開口器及健保卡讀寫卡設備等相關物品。\\

\color{OliveGreen}{照護內容}
& 基於安全考量,居家牙醫醫療服務時,以提供牙周病緊急處理、牙周敷料、牙結石清除、牙周暨齲齒控制基本處置、塗氟、非特定局部治療、特定局部治療、簡單性拔牙及單面蛀牙填補等服務為限。居家牙醫醫療服務給付項目及支付標準詳【附件17】。\\

\color{OliveGreen}{感染管制}
& 牙醫醫療服務應符合「牙醫巡迴醫療、特殊醫療、矯正機關之牙醫服務感染管制 SOP 作業細則」【附件14】。\\

\color{OliveGreen}{部分負擔}
& 依牙醫門診基本部分負擔計收。\\

\color{OliveGreen}{醫療費用}
& 特約醫事服務機構執行本計畫之醫療費用應按月申報給付項目【附件17】,並於門診醫療服務點數清單依下列規定填報「案件分類」及「特定治療項 目代號」欄位,案件分類 16,特定治療項目代號(一)請依病人類 別填報【極重度 FS、重度 FY、中度 L4、發展遲緩兒童 LE、失 能老人 L2、居整病人 LC、出院準備 LD、腦傷及脊髓損傷之中 度肢體障礙:LJ】。醫令類別「4 不得另計價之藥品、檢驗(查)、 診療項目或材料」。\\

\end{longtable}

\subsubsection{相關規範}
\begin{outline}
\1 個案首次接受居家牙醫醫療服務(含訪視)前,牙醫師須檢送申請
表【附件15】至牙醫全聯會,由牙醫全聯會於每月 20 日前將核可 名單函送保險人分區業務組備查。個案於首次接受居家牙醫醫療服 務(含訪視)後,須於次月 20 日前檢送病人之「口腔醫療需求評 估及治療計畫」【附件16】,正本送所屬保險人分區業務組、副本送 牙醫全聯會備查。經催繳三個月內仍未改善者,經保險人分區業務 組及牙醫全聯會確認,得暫停執行居家牙醫醫療服務。
\1 診療期間隨時注意病人之生理及心理狀況。若遇臨時緊急狀況或危急情形,應初步保護呼吸道,並立刻和計畫中最近的後送醫院聯絡,進行緊急醫療及後送程序。
\1 醫師應於院所製作電子病歷留存,且須將病人身分影印本及計畫所須之證明文件,黏貼於病歷首頁後掃瞄為電子檔 留存,以備查驗。
\1 設備之維護、清潔保養及醫療廢棄物由醫療院所依相關法規妥善處
理。

\end{outline}

\section{牙醫特殊計畫承辦窗口}
\noindent 社團法人中華民國牙醫師公會全國聯合會\\
牙醫特殊計畫承辦人朱智華\\
104台北市中山區復興北路420號10樓\\
電話:02-25000133 ext. 262\\
傳真:02-25000126\\
uase@cda.org.tw\\
% my project main text
% SWOT: Streghts, Weaknesses, Opportunities, Threats
% https://gist.github.com/ricardogarfe/4563453


\section{SWOT Matrix - \emph{(Strengths, Weaknesses, Opportunities, Threats)}}

% pentagon -> hexagon
\begin{tikzpicture}[
    hexagon/.style={%
        shape=regular polygon, regular polygon sides=6, minimum size=7.3cm, inner
        sep=-1mm, draw, fill=gray!10 %DarkSeaGreen!75!yellow
    }, font=\scriptsize\sffamily, thick
]

% \draw[help lines] (-16,-16) grid (16,16);
\filldraw[thin,gray,fill=gray!25] (-8,-8) rectangle (8,8);
\filldraw[thin,gray,fill=white] (-7.15,-7.15) rectangle (7.15,7.15);
\draw[thin,gray] (7.15,7.15)--(8,8) (-7.15,7.15)--(-8,8) (-7.15,-7.15)--(-8,-8)
(7.15,-7.15)--(8,-8);

% Strengths
% pentagon
\draw[thin, green] (-0.025,0.025)--(-7.05,0.025)--(-0.025,7.05)--cycle;

%\node[pentagon ,rotate=45] 
%\polygon[rotate=45]{6} 
\node[hexagon, rotate=45] at (-3.75,3.75) {
    \begin{varwidth}{\linewidth}
        \begin{itemize}[leftmargin=*,noitemsep]
            \item 社會責任與醫院評鑑
            %Technical and business expertise
            \item 守護文山新店深坑地區 %Domestic market orientation
            \item 偏鄉服務及特需中心 %初級
            %Stable management team
            \item 牙科部財務健全
            %Financial stability
            %\item Acquisition capabilities            
            \item 資深特需師資(四位) %七年30學分終身會員 %Economies of scale
            \item 複製北醫附醫五年到宅經驗 %Training programs
            %\item Loyalty and retention
        \end{itemize}
    \end{varwidth}
};
\draw (-2,2) node[rotate=45] {\large\textbf{Strengths}};

% Weaknesses
\draw[thin, brown!20!red] (0.025,0.025)--(7.05,0.025)--(0.025,7.05)--cycle;
\node[hexagon,rotate=-45] at (3.75,3.75) {
    \begin{varwidth}{\linewidth}
        \begin{itemize}[leftmargin=*,noitemsep]
            \item 初期經費與設備投資 %Centralized decisions
            \item  醫師、專責護士與人力規畫 %Accounts cross-selling
            \item 到宅牙醫行政文書
             %Marketing capabilities
            %\item 到宅醫療收入與時間成本 %Win on price image
            %\item 到宅交通費用 %BPO market
            %\item 特需案例來源
            %\item No differentiation
        \end{itemize}
    \end{varwidth}
};
\draw (2,2) node[rotate=-45] {\large\textbf{Weaknesses}};

% Opportunities
\draw[thin, green!50!blue] (-0.025,-0.025)--(-7.05,-0.025)--(-0.025,-7.05)--cycle;
%135
\node[hexagon,rotate=-45] at (-3.75,-3.75) {
    \begin{varwidth}{\linewidth}
        \begin{itemize}[leftmargin=*,noitemsep]
            \item  原PGY醫師(1人/月)支援連江縣立醫院 %Marketing push
            \item =>社區牙科訓練改為到宅牙醫服務 %Adding BPO capabilities
            \item 招募護士加入排班 %到宅醫師(二名排班) \item 到宅護士(二名排班)
            \item 牙科部祕書兼到宅牙醫行政
            \item 收取交通費/院方公務車支援 %Pricing structure
            \item 案例來源: 社區醫療部、出院準備服務 %Business process approach
            %\item %Annuity engagement
        \end{itemize}
    \end{varwidth}
};
\draw (-2,-2) node[rotate=-45] {\large\textbf{Opportunities}};

% Threats
\draw[thin, brown] (0.025,-0.025)--(7.05,-0.025)--(0.025,-7.05)--cycle;
\node[hexagon,rotate=45] at (3.75,-3.75) {
    \begin{varwidth}{\linewidth}
        \begin{itemize}[leftmargin=*,noitemsep]
            \item 到宅醫療收入不敷時間成本 %Profitability losses
            \item 收取到宅交通費形成門檻 %BPO market
            \item 特需案例來源
            \item 案主因病住院時,到宅服務暫停
            %\item High-risk deals
            %\item Image change inability
            %\item Degree of automation
        \end{itemize}
    \end{varwidth}
};
\draw (2,-2) node[rotate=45] {\large\textbf{Threats}};
%
\draw(0,-7.55) node {\Large EXTERNAL};
\draw(0,7.55) node {\Large INTERNAL};
\draw(-7.55,0) node[rotate=90] {\Large POSITIVE};
\draw(7.55,0) node[rotate=270] {\Large NEGATIVE};
\draw[green](-0.6,0.6) node {\Huge\textbf{S}}; 
\draw[brown!20!red](0.6,0.6) node {\Huge\textbf{W}};
\draw[green!50!blue](-0.6,-0.6) node {\Huge\textbf{O}};
\draw[brown](0.6,-0.6) node {\Huge\textbf{T}};
\end{tikzpicture}
  
  % treatment planning
% U:如何善用每個優勢? (How can we Use each Strength?)
%S:如何停止每個劣勢? (How can we Stop each Weakness?)
%E:如何成就每個機會? (How can we Exploit each Opportunity?)
%D:如何抵禦每個威脅? (How can we Defend against each Threat?)

臺北市立萬芳醫院牙科部特殊需求者牙科門診(特需中心)\\

\begin{outline}
% 零基預算 Zero-based budgeting system
% 成本中心
到宅牙醫經費概算:
\1 非專職人事(護士)月薪4.2萬元(每月到宅時數24/門診時數(160-24)=24/136=0.18),則到宅業務人事成本約為7412元; (依據勞基法「四週總工時不可超過160 小時」) %Required staffing to complete department functions; What staffing would be required at what experience levels & salaries
\1 設備(生理監測儀10萬元,攜帶型超音波洗牙機4萬元,攜帶型抽吸機0.5萬元) %Required equipment/software/hardware necessary to complete the function
\1 出診交通費(文山區200元,新店深坑400元,七堵暖暖700元) %Required overhead costs needed to complete the function
\1 耗材(每人次300元) %Required overhead costs needed to complete the function
\1 衛教宣導(禮品每次500元)
\1 教育訓練(講師鐘點費1000元)

\vspace{5mm}
預期效益及效益指標(key performance index, KPI): %Current marketing efforts and their efficiency
\1 每週出診數(每名醫師2次)、人次(首次家訪6,治療3)
\2 本計畫規範每位醫師執行特定需求者牙醫醫療服務及居家牙醫醫療服務(含評估訪視)合計每日達5人次以上,自第5人次起按6折支付,每日最多8人次為限,且每月以80人次為限;每位醫師支援每週不超過2日
\2 每人次健保收入(家訪1553元,治療5700元),交通費收入(200--700元)。
%,無轉診加成。
%(一)	預計提供高負荷家庭照顧者個案服務○○○人。
\2 每年預計提供家訪指導口腔照護及到宅治療及288人次/144人次
\2 KPI: 維持收案24名案主,每二個月穩定到宅治療一次
\2 每年健保收入小計1,268,064元
%(四)	預計拓展○○照顧資訊小站、○○照顧支持小站。
%(五)	預計辦理○○場教育訓練,受益人數為○○人次。
\1 預計辦理6場院內衛教宣導,擴及人數為60人次
\1 每年預計辦理長照家庭照顧者之口腔照護教育訓練2場,服務50人次

\end{outline}

2022年籌備期程(Gantt chart)\\
\begin{tabular}{|l|ccccccccc|}
\hline
 工作項目 & 四月 & 五月  & 六月 & 七月  & 八月  & 九月  & 十月  & 十一月  & 十二月  \\
\hline
 聘用護理人員    &  & &  & $\longrightarrow$ & $\longrightarrow$ & $\longrightarrow$ & $\longrightarrow$ & $\longrightarrow$ & $\longrightarrow$  \\
 設備採購    &  &  & $\longrightarrow$ & $\longrightarrow$ & &&&&  \\
 衛教及訓練    & $\longrightarrow$ & $\longrightarrow$ & $\longrightarrow$ & $\longrightarrow$  & $\longrightarrow$ & $\longrightarrow$ & $\longrightarrow$ & $\longrightarrow$ & $\longrightarrow$ \\
 到宅服務    &  & &  & $\longrightarrow$ & $\longrightarrow$ & $\longrightarrow$ & $\longrightarrow$ & $\longrightarrow$ & $\longrightarrow$  \\
 到院服務*  & $\longrightarrow$ $\longrightarrow$ & $\longrightarrow$ & $\longrightarrow$ & $\longrightarrow$  & $\longrightarrow$ & $\longrightarrow$ & $\longrightarrow$ & $\longrightarrow$ & $\longrightarrow$ \\ 
\hline
\end{tabular}\\
(*原已執行中的業務)

\clearpage
%\vspace{5mm}
區域內整體服務辦理情形 or
身心障礙牙科醫療服務網絡模式
特殊需求者牙科醫療服務轉介單(附件2-4)
%(盤點申請區域服務資源及說明單位組織量能、目前長照服務推動情形外,請加強敘明目前及未來辦理家庭照顧者服務情形): 社區資源開發整合
%https://www.nhi.gov.tw/BBS_Detail.aspx?n=73CEDFC921268679&sms=D6D5367550F18590&s=DC5108C6E3DD21D0

依據公告特需中心名單(2022年)
\begin{outline}
%\0 

\1 到宅牙醫服務
\2 全國136名醫師
\2 臺北市52名醫師
    \3 臺北醫學大學附設醫院4名醫師
    \3 文山區計有德威牙醫診所2名醫師
    %陳義聰、蕭雅純二名醫師
\2 新北市: 雙和醫院8名醫師

\1 特需服務院所(到院牙醫服務)
\2 文山區計有萬芳醫院及13家診所,擬加強社區照護網絡,視案主病情須要,互相轉診,未來將由萬芳醫院及德威牙醫診所提供到宅牙醫服務
\end{outline}

臺北市文山區特須牙醫院所名單\\
\begin{tabularx}{1.0625\textwidth}{|c|p{3.2cm}|c|l|}
\hline
臺北&	童芯牙醫診所&	02-29366707 &	臺北市文山區久康街24巷7號1樓\\
\hline
臺北&	全家牙醫診所&	02-22346393 &	臺北市文山區木柵路2段143號\\
\hline
臺北 &	欣美牙醫診所 &	02-22343708 &	臺北市文山區木柵路3段10號(1、2樓)\\
\hline
臺北 &	永麗牙醫診所 &	02-29372227 &	臺北市文山區木新路3段111號(1、2樓)\\
\hline
臺北 &	天丞牙醫診所 &	02-29378380 &	臺北市文山區木新路3段325號\\
\hline
臺北 &	驊陽牙醫診所 &	02-86610188 &	臺北市文山區忠順街1段26巷32號\\
\hline
臺北 &	景華牙醫診所 &	02-29304167 &	臺北市文山區景華街122號1樓\\
\hline
臺北 &	德在牙醫診所 &	02-29338248 &	臺北市文山區景福街30號1樓\\
\hline
臺北 &	雅田牙醫診所 &	02-29352838 &	臺北市文山區景興路119號1樓\\
\hline
臺北 &	家樂牙醫診所 &	02-82301197 &	臺北市文山區萬安街33號\\
\hline
臺北 &	*德威牙醫診所 &	02-29328281 &	臺北市文山區興隆路1段72、74號、70巷1-1號2樓\\
\hline
臺北 &	臺北市立萬芳醫院-委託財團法人臺北醫學大學辦理
&	02-29307930 &	臺北市文山區興隆路3段111號\\
\hline
臺北 &	廣泉牙醫診所 &	02-29343572 &	臺北市文山區興隆路3段41號\\
\hline
臺北 &	德全牙醫診所 &	02-29341569 &	臺北市文山區羅斯福路6段248號1樓\\
\hline
\end{tabularx}\\
(*已提供到宅牙醫服務 \today)

% https://ceriniandassociates.com/news-feed/2020/07/13/zero-based-budgeting/
%One tool which can be helpful to practices of all sizes can be to periodically implement a “zero-based” budgeting approach. Zero-based budgeting was first developed and implemented in the 1970s and then driven to extremes by investment firms such as 3G Capital in the 2000s as a tool to cut costs wherever possible and to extremes, such as focusing on such minute details as the number of pages printed and photocopies made by employees. As a result, zero-based budgeting has a reputation as an “austerity” measure and a tool used to cut costs to the bone. This zealous approach to the ideals may have its place in many organizations, however, the original ideals and principles of the system should be part of any organization’s planning toolbox from time-to-time.

%In a “zero-based” budgeting system, departments should look periodically at each year and budget as if starting from a zero-dollar budget allocation, rather than just what was spent in the past. The real point of the exercise is to take a top-down approach and determine if all spend in any given department is required to fulfill the functions of this department. The additional scrutiny can be used to uncover potential inefficiencies, over or understaffing in departments, discover potential synergies between departments, and empower department heads to perform an overall review of their operations.

%In implanting a zero-based budgeting system, a healthcare organization should take the following steps:
%1.) Start a baseline zero for all departments; prior-year spending does not matter.
%2.) Evaluate every cost area within the department (in conjunction with the department head). This evaluation should include:
%a.) Required staffing to complete department functions
%b.) Required equipment/software/hardware necessary to complete the function
%c.) Required overhead costs needed to complete the function

%When evaluating, consider how one would start *** a brand-new department from scratch. 
%What staffing would be required at what experience levels & salaries; what costs are required to perform the necessary operations? 
%The critical eye here is necessary and often best done in a collaborative effort with both an insider (someone in the department) and an outsider (someone outside the department). 外部委員

%3.) Justify the spending in these above areas and try to identify cost savings. Some examples may include:
%a.) Looking at the cost of outsourcing billing vs. proving billing services in-house
%b.) Staffing levels of reception/administrative staff
%c.) Current marketing efforts and their efficiency


%\bibliographystyle{plainnat}
%\bibliography{ref}

\newpage
\appendix
%\input{content/appendix}
\includepdf[pages=-]{111年牙醫特殊醫療服務計畫 (ICF).pdf}
%x\includepdf[pages=-]{111年牙醫特殊醫療服務計畫 (附件3).pdf}
\includegraphics[width=0.9\textwidth]{111年牙醫特殊醫療服務計畫 (附件3).png}
\clearpage
%\includepdf[pages=-]{111年牙醫特殊醫療服務計畫 (附件7).pdf}
\includegraphics[width=0.9\textwidth]{111年牙醫特殊醫療服務計畫 (附件7).png}
\includepdf[pages=-]{111年牙醫特殊醫療服務計畫 (附件12).pdf} % 附件12 診紀錄
\includepdf[pages=-]{A040170081040500-1090331-7000-014.pdf} % 附件14
\includepdf[pages=-]{111年牙醫特殊醫療服務計畫 (DOH_附件).pdf}

% Please add the following required packages to your document preamble:
% \usepackage{graphicx}

【附件19】

\begin{table}[H]
%\caption{}
%\label{tab:DOH_acls}
\centering
\resizebox{\textwidth}{!}{%
\begin{tabular}{|l|l|l|l|}
\hline
分類       & 名稱                           & 功能                                                                   & 備註      \\ \hline
BLS &
  Vital sign monitor &
  \begin{tabular}[c]{@{}l@{}}NIBP blood pressure (sphygmomanometer),\\ pulse oximeter (oxygen saturation, SpO2), pulse rate, BPM\end{tabular} &
  HP/Phillip \\ \hline
       & Glucose Meter                & blood sugar from finger tip                                          &         \\ \hline
       & Stethoscope                  & 心臟科聽診器                                                               &         \\ \hline
       & Oxygen supply                & ambu bag, face mask 6L/min; nasal prong (with cannula)               & O2 tank \\ \hline
       & Airway                       & nasopharyngeal airway (nasal trumpet, size #7 #8)                    &         \\ \hline
 &
  Airway (advanced) &
  \begin{tabular}[c]{@{}l@{}}laryngeal mask airway (LMA), oral endotracheal tube (French #6, #7)\\ laryngoscope (插管用喉鏡 curved blade)/Magi forceps + Lidocaine jelly\end{tabular} &
   \\ \hline
Drug   & Allergy with broncospasm     & 肌肉內注射 0.2-0.5mg epinephrine (1 mg/mL)                                &         \\ \hline
       & Status epilepticus (> 5 min) & diazepam 肌肉或靜脈注射(靜脈注射較佳),初劑量 5-10mg                                  &         \\ \hline
Dental & 攜帶式超音波洗牙機                    & NSK Varios 970 LUX (water supply, fibroptic cable)                   &         \\ \hline
       & 攜帶式真空吸引機(suction machine)    & 抽痰管, suction tip with saliva ejector, 治療巾, 彎盆                        &         \\ \hline
       & PPE                          & 口罩surgical mask 隔離衣gown 護目鏡goggle                                    & x3      \\ \hline
       & Fluoridation                 & 氟膠 Sodium Fluoride gel 5\%(Clinpro[3M] vanish with brush), 2x2 gauze &         \\ \hline
 &
  衛教 &
  \begin{tabular}[c]{@{}l@{}}張口棒(健康牌)、木質壓舌板、\\ 兒童牙刷(soft brush, flat handle)、牙間刷、3M牙線棒\end{tabular} &
   \\ \hline
       & 感控                           & 75\% alcohol swab、乾洗手液,檢診手套,紅色感染用塑膠袋                                 &         \\ \hline
Others &                              & 手電筒, 電源延長線/adaptor                                                   &         \\ \hline
\end{tabular}%
}

\end{table} % 附件19

\end{document}



%%% a list in TMWH
到宅牙醫醫療服務已納入「全民健康保險居家醫療照護整合計畫」中
%https://cda.org.tw/cda/public_medical_institution_type_b_search_result.jsp

02-29307930*7088王秘書 116 台北市文山區興隆路3段111號2樓轉牙科
台北市立萬芳醫院-委託財團法人私立台北醫學大學辦理

王柏翔
卓郁純
V林光勳
翁瑛祺
許晶晶
陳育德
V陳俊銘
V陳培惠
黃培琪
V黃曉楓
蔡昀潔
V謝承祐 


%%%%%%%%%%%
%%%%%%%%%%%

% pulse oximeter 金手指測血氧, 
%lead II ECG (白黑紅三電極貼片)
NIBP blood pressure (sphygmomanometer),
pulse oximeter (oxygen saturation, SpO2), pulse rate, BPM) (HP/Phillip)
%AED (Philips)

Glucose Meters

stethoscope心臟科聽診器,
ambu bag,
face mask 6L/min; nasal prong (with cannula) + O2 tank,
oral endotracheal tube (French #6, #7)
laryngoscope (插管用喉鏡 curved blade)/Magi forceps + Lidocaine jelly => 與兩支木舌壓用一條治巾包在一起,
nasopharyngeal airway (nasal trumpet, 長度 #7 #8), 抽痰管
laryngeal mask airway (LMA)
18 gauge needle

IV line + Normal saline 250cc x2,

% ppe
%mask, surgical glove, goggle, gown(隔離袍)

% Dental 
攜帶式超音波洗牙機(NSK Varios 970 LUX (water supply, light cable)
攜帶式真空吸引機(suction machine), suction tip with saliva ejector)
治療巾, 彎盆

口罩surgical mask 隔離衣gown 護目鏡goggle (二套), %surgical glove (#7.5; #7.0)
2x2-inch gauze
氟膠 Sodium Fluoride gel 5\%(Clinpro[3M] vanish with brush) (P30002)
%mepevacaine 3\% (local anesthesia)
%gelfoam/surgicel with suturing
張口棒(健康牌)
木質壓舌板
兒童牙刷(soft brush, flat handle),牙間刷、3M牙線棒

75\% alcohol swab
乾洗手液,檢診手套
手電筒, 電源延長線/adaptor
紅色感染用塑膠袋

% ACLS drugs


[brief summary of ALCS/CPR 密訣]

=toxicity: Na blocker (lidocaine=> heart block (wide QRS, asystole , brain block: BZD for seizure)
intralipid therapy

=allergy of local anaesthesia, latex, or muscle relaxant => antigen IgE => bronchospasm, strider, shock
Rx: epinephrine IM 0.5mg, then IV fluid (2,000mL)
IH bronchodilator
IV corticosteroid or anti-histamine
首要需立即給予肌肉內注射0.2-0.5mg epinephrine (1 mg/mL),再依病人狀況後續給予藥效較緩慢的類固醇及抗組織胺類藥物治療
維持呼吸道airway、呼吸breathing、血液循環circulation及 checking BP/HR/RR
(SpO2)

=bronchospasm (asthma)
near fatal asthma
wheezing, SpO2 lowering, CO2 retention
RX: O2, beta-2 agonist, dexamethasone 10mg
anti-cholinergic ipratropium
epinephrine SC 0.3mg *3 Q20mins

=laryngospasm
reflex of glottis of cricothyroid muscle by stimulus of superior laryngeal nerve
stridor:
=> intubation under succinylcholine
500mg/vial
dilution in 10mL N/S

=Seizure: RX: BZD, propofol, barbiturate